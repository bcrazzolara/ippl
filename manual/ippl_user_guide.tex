\documentclass[12pt,a4paper]{report}
%% does not work in Latex2Html mode
%\usepackage{hyperref}
\usepackage{url}
\usepackage{html}
\usepackage{epic}
\usepackage{eepic}
\usepackage{makeidx}
\usepackage{array}
\usepackage{times}
\usepackage{amsmath}
\usepackage[dvips]{epsfig,rotating}
\usepackage{dingbat}
\usepackage{listings}
\usepackage{graphicx}

\newcommand{\deprecated}[1]{\textcolor{red}{#1}}
\newcommand{\changed}[1]{\textcolor{green}{#1}}

\lstnewenvironment{code}[1][]
{\textbf{Code Listing} \hspace{1cm} \hrulefill \lstset{language=C++, basicstyle=\ttfamily\scriptsize}}
{\hrule\smallskip}
%{\textbf{Code Listing} \hspace{1cm} \hrulefill \lstset{language=C++, basicstyle=\ttfamily\scriptsize, #1}}

\lstnewenvironment{codeln}[1][]
{\textbf{Code Listing} \hspace{1cm} \hrulefill \lstset{language=C++, basicstyle=\ttfamily\scriptsize, numbers=left, numberstyle=\tiny, stepnumber=1, numbersep=5pt}}
{\hrule\smallskip}

\lstnewenvironment{smallcode}[1][]
{\lstset{language=C++, basicstyle=\ttfamily\scriptsize}}
{\smallskip}

\usepackage{pdfdraftcopy}             % Draft
\draftstring{V 2.0 - with Kokos}
%% Define a new 'leo' style for the package that will use a smaller font.
%\makeatletter
%\def\url@leostyle{%
 % \@ifundefined{selectfont}{\def\UrlFont{\sf}}{\def\UrlFont{\small\ttfamily}}}
%\makeatother
%% Now actually use the newly defined style.
%\urlstyle{leo}

\setlength\topmargin{0.0cm}
%\setlength\oddsidemargin{3.501cm}
%\setlength\evensidemargin{3.501cm}
\renewcommand{\arraystretch}{1.5}


% Definition neuer Zeichen
\newcommand{\R}{{\mathbb R}}
\newcommand{\N}{{\mathbb N}}
\newcommand{\Z}{{\mathbb Z}}
% Standard Notationen
\newcommand{\vc}[1]{\mbox{\boldmath $#1$}}
\newcommand{\sv}[1]{\mbox{\boldmath $\vec{#1}$}}
\newcommand{\mx}[1]{\mbox{\boldmath $\underline{#1}$}}
\newcommand{\op}[1]{\mathcal{#1}}
% Sonstige Abkuerzungen
\newcommand{\backpar}{\rule{1mm}{0mm}\\[-5ex]\rule{1mm}{0mm}}
\newcommand{\ds}{\displaystyle}
\newcommand{\AND}{\quad\mbox{and}\quad}
\newcommand{\diag}{\mbox{diag }}
\newcommand{\arrstr}[1]{\rule[0mm]{0mm}{#1}}
\newcommand{\hstrut}[1]{\rule{0mm}{#1}}
\newlength{\mylablen}
\newlength{\mycaplen}
\newlength{\myislen}
% caption fuer Bilder
\newcommand{\fcap}[2]{\refstepcounter{figure} \settowidth{\mycaplen}{#2}
\settowidth{\mylablen}{\bf Figure \thefigure:~}
\settowidth{\myislen}{\textwidth-\mylablen}
\begin{center}
{\bf Figure \thefigure:~}\ifthenelse{\mycaplen > \myislen}{\parbox[t]{\textwidth-\mylablen}{#2}\label{#1}}{#2\label{#1}}\end{center}}
% caption fuer Tabellen
\newcommand{\tcap}[2]{\refstepcounter{table} \settowidth{\mycaplen}{#2}
\settowidth{\mylablen}{\bf Table \thetable:~}
\settowidth{\myislen}{\textwidth-\mylablen}
\begin{center}
{\bf Table \thetable:~}\ifthenelse{\mycaplen > \myislen}{\parbox[t]{\textwidth-\mylablen}{#2}\label{#1}}{#2\label{#1}}\end{center}}
%
\newtheorem{theorem}{Theorem}
\newtheorem{lemma}{Lemma}
\newtheorem{corollary}{Corollary}
\newtheorem{remark}{Remark}
\newtheorem{definition}{Definition}
\newtheorem{proposition}{Proposition}
\newenvironment{proof}{{\bf Proof:}}{{}\hfill{$\square$}\par\smallskip}

\renewcommand{\topfraction}{0.9}
\renewcommand{\bottomfraction}{0.9}
\renewcommand{\textfraction}{0.1}

\newcommand{\ipplversion}{\text{2.0 }}
\newcommand{\mad}{\textsc{mad }}
\newcommand{\madnine}{\textsc{mad9 }}
\newcommand{\madninep}{\textsc{mad9p }}
\newcommand{\madeight}{\textsc{mad8 }}
\newcommand{\pooma}{\textsc{pooma }}
\newcommand{\classic}{\textsc{classic }}

\newcommand{\ippl}{\textsc{I$\text{P}^2$L }}


\newcommand{\opal}{\textsc{OPAL }}
\newcommand{\opalt}{\textsc{OPAL-t }}
\newcommand{\opalcycl}{\textsc{OPAL-cycl }}
\newcommand{\opalmap}{\textsc{OPAL-map }}

\newcommand{\partroot}{\textsc{H5PartROOT }}

\newcommand{\lieop}[1]
{{:}{#1}{:}}
\renewcommand{\arraystretch}{2.0}

\newcommand{\rms}[1]
{\overset{\sim}{#1}}

% Pfister stuff
\def\C{{{{\rm {\mbox{\small l}}} \kern -.50em {\rm C}}}}
\def\R{{{{\rm l} \kern -.15em {\rm R}}}}
\def\N{{{{\rm l} \kern -.15em {\rm N}}}}
\def\E{{{{\rm l} \kern -.15em {\rm E}}}}
\def\P{{{{\rm l} \kern -.15em {\rm P}}}}
\def\D{{{{\rm l} \kern -.15em {\rm D}}}}
\def\L{{{{\rm l} \kern -.15em {\rm L}}}}
\def\Z{{{{\rm Z} \kern -.35em {\rm Z}}}}
\def\Q{{{{\rm {\mbox{\small l}}} \kern -.45em {\rm Q}}}}
\def\qQ{{{{\rm {\mbox{\scriptsize l}}} \kern -.33em {\rm Q}}}}
\def\v{\vspace{1ex}}
\def\-v{\vspace{-1ex}}
\def\vv{\vspace{2ex}}
\def\vvv{\vspace{3ex}}
\def\vn{\vspace{3ex}\noindent}
\def\hs{\hspace{-1ex}}
\def\n{\noindent}
\def\dsl{\dis\sum\limits}
\def\dis{\displaystyle}
\def\O{\Omega}
\def\o{\omega}
\def\fr{\mbox{\footnotesize $\dis\frac{1}{2}$}}
\def\frvier{\mbox{\footnotesize $\dis\frac{1}{4}$}}
\def\ov{\overline}
\def\b{\big[}
\def\B{\big]}
\def\i{\big\{}
\def\g{\big\}}
\def\e{\epsilon}
\def\f{\footnotesize}
\def\l{\ell^\prime}
\def\ll{\ell,\ell^\prime}
\def\r{\rightarrow}
\def\point{{\mbox{\large $.$}}}
\def\wl{\widetilde{\ell}}
\def\wll{\widetilde{\ell},\widetilde{\ell}^\prime}
\def\w{\widetilde}
\def\wh{\widehat}
\def\wt{\widetilde}
\def\cA{{\cal A}}
\def\cT{{\cal T}}
\def\cE{{\cal E}}
\def\oS{\overline{S}}
\def\oT{\overline{T}}
\def\cI{{\cal I}}
\def\cK{{\cal K}}
\def\cH{{\cal H}}
\def\cP{{\cal P}}
\def\cF{{\cal F}}
\def\cS{{\cal S}}
\def\cG{{\cal G}}
\def\cO{{\cal O}}
\def\cU{{\cal U}}

%% REAL ADA STUFF
\def\bx{{\bf x}}
\def\br{{\bf r}}
\def\bv{{\bf v}}
\newcommand{\matr}[1]{\mathcal{#1}}
 \newcommand{\myvec}[1]{\bf{#1}}
%\newcommand{\figref}[2]{#1 (see~Figure~\ref{fig:#2})}
%\newcommand{\tabref}[2]{#1 (see~Table~\ref{tab:#2})}
%\newcommand{\secref}[2]{#1 (see~Section~\ref{sec:#2})}
\newcommand{\mytabline}[2]{\texttt{#1} & #2 \\}
\renewcommand{\vec}{\myvec}

\makeindex

\renewcommand{\topfraction}{1.0}
\renewcommand{\bottomfraction}{1.0}
\renewcommand{\textfraction}{0.0}

\newenvironment{tex2html_nowrap}{}{}

\newcommand{\bibref}[2]{\hypercite{#1}{}{}{bib:#2}}

\newcommand{\tabline}[3]{\hyperref{\texttt{#1}}%
  {\texttt{#1} (}{)}{sec:#3} & #2 \\}
\newcommand{\figref}[2]{\hyperref{#1}{#1 (see~Figure~}{)}{fig:#2}}
\newcommand{\secref}[2]{\hyperref{#1}{#1 (see~Section~}{)}{sec:#2}}
\newcommand{\tabref}[2]{\hyperref{#1}{#1 (see~Table~}{)}{tab:#2}}
\begin{htmlonly}
\bodytext{bgcolor = "#ffffff"}
\end{htmlonly}

%\newcommand{\documentlabel}{ }

\begin{document}

\begin{titlepage}

\begin{htmlonly}
\begin{rawhtml}
<center>
PAUL SCHERRER INSTITUT
<br>
<h1>The IPPL Framework , Version 2.0</h1>
<h2>(Independent Parallel Particle Layer)</h2>
<h2>User's Reference Manual</h2>
<br>
Andreas Adelmann
</center>
\end{rawhtml}
\end{htmlonly}

\begin{latexonly}
\begin{center}\normalsize
\includegraphics[width=1.\linewidth,angle=0]{figures/psi_logo_wide}
\end{center}
\vskip 0.7cm
PSI-PR-09-05
\begin{flushright}
%\documentlabel \\                    % document label
\end{flushright}
%\vskip 2.3cm
\begin{center}\LARGE                 % document title
{\bf The \ippl Framework} \\
(Independent Parallel Particle Layer) \\
Version \ipplversion \footnote{Release Date: \today} \\
{\bf User's Reference Manual}
\end{center}
\vskip 1.5em
\begin{center}                       % author
Andreas Adelmann, ..... \vskip 2em
{\large Abstract}
\end{center}
\end{latexonly}
\begin{quotation}
 \ippl (Independent Parallel Particle Layer) is an object-oriented framework for particle based applications in computational science
requiring high-performance parallel computers. One of \ippl 's most attractive features is its high performance on both single-processor and  distributed-memory multicomputers machines. \deprecated{As future releases of the library will also support shared-memory multicomputers, \ippl 's authors have had to think very carefully about how to obtain the best possible performance across a wide range of applications on different architectures.}
\ippl  is a library of C++ classes designed to represent common abstractions in applications where {\em particles, fields} and operators like {\em FFT's} are needed.

Application programmers use and derive from these classes, which present a data-parallel programming interface at the highest abstraction layer.

Lower, to the user (programmer) hidden implementation layers encapsulate distribution and communication of data among processors. \deprecated{The supported platforms are: Linux based Beowulf clusters, Cray XT5/6, SGI Ultrix and the IBM Blue Gene series.}

The main goals of the \ippl  framework includes:
\begin{itemize}
    \item Portability across serial, distributed, and parallel architectures with no change to source code
    \item Development of reusable, cross-problem-domain components to enable rapid application development
    \item High efficiency for kernels and components relevant to scientific simulation
    \item Framework design and development driven by applications from a diverse set of scientific problem domains
    \item Shorter time from problem inception to working parallel simulations
\end{itemize}
 \ippl is currently in development the version here is the first "developer release". \ippl is inspired and partially based on POOMA r1 designed and implemented by scientists working at the Los Alamos National Laboratory's Advanced Computing Laboratory.
 %Although \ippl  has received numerous additions the most importand are:
%Fully ANSI C++ compatible, full 64Bit, HDF5 support and a Trilinos interface.

The report is organized as follow: in chapter 1 an introduction based on examples including installation instructions is presented. Chapter 2 and 3 are describing support classes followed by an discussion
on 3D parallel FFT's in \ippl. The next two chapters explaining the use of particles and fields. Appendix A - F describing design and implementation details of the most important classes in the
framework.
\end{quotation}
\vfill


\end{titlepage}

\tableofcontents
\listoftables
\listoffigures
\chapter{Introduction}
\label{sec:Introduction}
One of \ippl 's most attractive features is its high performance on both single-processor and distributed-memory multiprocessor machines. \deprecated{As future releases of the library will also support shared-memory machines.}

\deprecated{The heart of the problem \ippl 's authors face is that while data-parallel programming is a natural way to express many scientific and numerical algorithms, straightforward implementations of it do exactly the wrong thing on modern architectures, whose performance depends critically on the re-use of data loaded into cache. If a program evaluates A+B+C for three arrays A, B, and C by adding A to B, then adding C to that calculation's result, performance suffers both because of the overhead of executing two loops instead of one, but also (and more importantly) because every value in the temporary array that stores the result of A+B has to be accessed twice: once to write it, and once to read it back in. As soon as this array is too large to fit into cache, the program's performance will drop dramatically.}

%The first section of this tutorial explains what \ippl  does to solve this problem. Subsequent sections discuss other advanced aspects of \ippl , such as how to build pointwise functions, or reduction functions that will execute efficiently regardless of how arrays are laid out in memory.

%\ippl  tries to resolve the tension between how programmers want to express their ideas, and what runs most efficiently on modern architectures, by delaying the evaluation of expressions until enough is known about their context to ensure that they can be evaluated efficiently. It does this by blocking calculations into units called iterates, and putting those iterates into a queue of work that is still to be done. Each iterate is a portion of a computation, over a portion of a domain. \ippl  tracks data dependencies between iterates dynamically to ensure that out-of-order computation cannot occur.

%Depending on the switches specified during configuration when the library is installed, and the version of the \ippl  library that a program is linked against, \ippl  will run in one of four different modes. In the first mode, the work queue doesn't actually exist. Instead, the single thread of execution in the program evaluates iterates as soon as they are "enqueued", i.e. work is done immediately. The result is that all of the calculations in a statement are completed by the time execution reaches the semi-colon at the end of that statement.

%In its second mode, \ippl  maintains the work queue, but all work is done by a single thread. The queue is used because the explicit parceling up of work into iterates gives \ippl  an opportunity to re-order or combine those iterates. While the overhead of maintaining and inspecting the work queue can slow down operations on small arrays, it makes operations on large arrays much faster.

%For example, consider the four function calls that perform red/black relaxation in the second tutorial. In order to get the highest possible performance out of the cache, all four of these expressions should be evaluated on a given cache block before any of the expressions are evaluated for the next cache block. Managing this by hand is a nightmare, both because cache size varies from machine to machine (even when those machines come from the same manufacturer), and because very slight changes in the dimensions of arrays can tip them in or out of cache. \ippl 's array classes and overloaded operators do part of the job by creating appropriately-sized iterates; its work queue does the rest of the job by deciding how best to evaluate them. The net result is higher performance for less programmer effort.

%\ippl 's third and fourth modes of operation are multi-threaded. Each thread in a pool takes iterates from the work queue when and as they become available. Iterates are evaluated independently; the difference between the two modes is that one is synchronous, and blocks after evaluating each data-parallel statement, while the other is asynchronous, i.e. permits out-of-order execution. The table below summarizes these four modes, along with the configuration arguments used to produce each.


\section{\deprecated{Example 1 Laplace solver using Jacobi iteration}}
\begin{code}[caption={Laplace solver}]
#include "Ippl.h"
int main(int argc, char *argv[])
{
    Ippl ippl(argc,argv);
    Inform msg(argv[0]);
    const unsigned N=8;
    const unsigned Dim=2;

    Index IGLOBAL(N);  // Specify the global domain
    Index JGLOBAL(N);

    Index I(1, N-1); // Specify the interior domain
    Index J(1, N-1);
    FieldLayout<Dim> layout(IGLOBAL,JGLOBAL);
    GuardCellSizes<Dim> gc(1);
    typedef UniformCartesian<Dim> Mesh;
    Field<double,Dim,Mesh> A(layout,gc);
    Field<double,Dim,Mesh> b(layout,gc);

    assign(A,0.0);  // Assign initial conditions
    assign(b,0.0);

    b[N/2][N/2] = -1.0;  // put a spike on the RHS
    double fact = 0.25;

    // Iterate 200 times
    for (int i=0; i<200; ++i) {
        assign(A[I][J],fact*(A[I+1][J] +
                             A[I-1][J] +
                             A[I][J+1] +
                             A[I][J-1] - b[I][J]));
    }
    msg << A << endl;
    return 0;
}
\end{code}
The syntax is very similar to that of Fortran 90: a single assignment fills an entire array with a scalar value, subscripts express ranges as well as single points, and so on. In fact, the combination of C++ and \ippl provides so many of the features of Fortran 90 that one might well ask whether it wouldn't better to just use the latter language straight up. One answer comes down to economics. While the various flavors of Fortran are still used in scientific computing, Fortran's user base is shrinking, particularly in comparison to C++. Networking, graphics, database access, and operating system interfaces are available in C++ programmers long before they're available in Fortran (if they become available at all). What's more, support tools such as debuggers and memory inspectors are primarily targeted at C++ developers, as are hundreds of books, journal articles, and web sites.

Another answer is that the abstraction facilities of C++ are much more powerful that those in Fortran. While Fortran 90 supports an attractive array syntax for floating point arrays one could not, for example, efficiently extend this high level syntax to arrays of vectors or tensors. Until recently, Fortran has had two powerful arguments in its favor: legacy applications, and performance. However, the importance of the former is diminishing as the invention of new algorithms force programmers to rewrite old codes, while the invention of techniques such as {\it expression templates} has made it possible for C++ programs to match, or exceed, the performance of highly-optimized Fortran 77.
%!TEX encoding = UTF-8 Unicode

\section{\deprecated{Example 2 Power Spectrum}}
A sinussoidal field  $\rho(i,j,k) = a_1sin(k_1 \frac{2\pi}{n_x} i) + a_5 sin(k_5 \frac{2\pi}{n_x} i)$,
$i= 1 \dots n_x, ~ j= 1 \dots n_y, ~ k= 1 \dots n_z $ with $n_x,n_y$ and $n_z$ denoting the grid size is generated and
the power spectrum calculated. This examples shows how to initialise fields, compute discrete complex-complex FFT and
compute the resulting powerspectrum.

Assume a real density field is defined like
\begin{smallcode}
typedef Field<double,Dim,Mesh_t,Center_t>  Field_t;
Field_t rho;
\end{smallcode}
we then can immediately initialize the field according to the above formula
\begin{smallcode}
assign(rho[I][J][K], a1*sin(2.0*pi/nr_m[0]*k1*I) +
                     a5*sin(2.0*pi/nr_m[0]*k5*I));
\end{smallcode}
Normalizing to $\max(\rho) \le 1.0 $ with
\begin{smallcode}
rho /= max(rho)
\end{smallcode}
we then assume to have defined a complex field "fC" and a complex-complex FFT.
\begin{smallcode}
fC = rho;
fft->transform("forward" , fC);
 \end{smallcode}
Here we used the in place version of the FFT to obtain $\rho $ in Fourier space. Now
we can compute the power spectrum:
\begin{smallcode}
pwrSpec = real(fC*conj(fC));
\end{smallcode}
%\pagebreak
and calculate the 1D pwr-spectrum (in x direction) by integrating over y and z: \\
\begin{code}
 NDIndex<3> elem;
 for (int i=lDomain[0].min(); i<=(lDomain[0].max()-1)/2; ++i) {
  elem[0]=Index(i,i);
   for (int j=lDomain[1].min(); j<=(lDomain[1].max()-1)/2; ++j) {
    elem[1]=Index(j,j);
     for (int k=lDomain[2].min(); k<=(lDomain[2].max()-1)/2; ++k) {
       elem[2]=Index(k,k);
       f1D[i] += pwrSpec.localElement(elem);
     }
   }
 }
\end{code}
The power spectra of the local domain is stored in $f1$. We have to update all other node
so that each node has the full power spectrum by:
\begin{smallcode}
reduce(&(f1[0]),&(f1[0])+f1_lenght,OpAddAssign());
\end{smallcode}
assuming the non local part of $f1$ is initialized with zero.

%The full code pwrspec-1.cpp is located at $\ippl_ROOT/test/simple.


 \section{\deprecated{Example 3 Particle in Cell Code (PIC)}}
 This example discusses how to write a 3D Particle in Cell Code (PIC). The
 complete source file can be found at {\em \$IPPL\_ROOT/test/particles}. The
 this presentation details are omitted, only the structure and important issues are
 highlighted.
 \subsection{\deprecated{The {\em ChargedParticles} Class}}
 The base class {\tt ParticleBase} is augmented with attributes such as charge to mass ration
 {\tt qm}, the vector momenta {\tt P} and the vector holding the electric field {\tt E}.
 \begin{code}
 ChargedParticles(PL* pl, Vector_t hr, Vector_t rmin,
                  Vector_t rmax, e_dim_tag decomp[Dim]) :
                  ParticleBase<PL>(pl),
                  hr_m(hr),
                  rmin_m(rmin),
                  rmax_m(rmax),
                  fieldNotInitialized_m(true)
{
    this->addAttribute(qm);
    this->addAttribute(P);
    this->addAttribute(E);

    for (int i=0; i < 2*Dim; i++) {
        this->getBConds()[i] = ParticlePeriodicBCond;
        bc_m[i]  = new PeriodicFace<double  ,Dim,Mesh_t,Center_t>(i);
        vbc_m[i] = new PeriodicFace<Vector_t,Dim,Mesh_t,Center_t>(i);
    }
    for(int i=0; i<Dim; i++)
        decomp_m[i]=decomp[i];
}
\end{code}
The arrays {\tt bc\_m} and {\tt vbc\_m} holding the boundary conditions for particles and fields.
In {\tt decomp\_m} the domain decomposition is stored.

\subsection{\deprecated{The {\em main}}}

 \begin{code}
int main(int argc, char *argv[]) {
    Ippl ippl(argc, argv);
    Inform msg(argv[0]);

    Vektor<int,Dim> nr(atoi(argv[1]),atoi(argv[2]),atoi(argv[3]));

    const unsigned int totalP = atoi(argv[4]);
    const int nt              = atoi(argv[5]);

    e_dim_tag decomp[Dim];
    int serialDim = 2;

    Mesh_t *mesh;
    FieldLayout_t *FL;
    ChargedParticles<playout_t>  *partBunch;

    NDIndex<Dim> domain;
    for(int d=0; d<Dim; d++) {
        domain[d] = domain[d] = Index(nr[d] + 1);
        decomp[d] = (d == serialDim) ? SERIAL : PARALLEL;
    }
\end{code}
In the fist part of main, the discrete computational domain ({\tt domain}) and the
domain decomposition ({\tt decomp}) is constructed. We have choose a 2D domain decomposition
with $z$ serial i.e. not parallelized. \\
\begin{code}
    mesh          = new Mesh_t(domain);
    FL            = new FieldLayout_t(*mesh, decomp);
    playout_t* PL = new playout_t(*FL, *mesh);

    Vector_t hr(1.0);
    Vector_t rmin(0.0);
    Vector_t rmax(nr);

    partBunch=new ChargedParticles<playout_t>(PL,hr,rmin,rmax,decomp);
\end{code}
 Here we construct the {\tt mesh} the field layout ({\tt FL}), describing how the fields are distributed
  and finally the particle layout {\tt PL}. The latter is the used as a template argument to construct the
  particle container. For this example the mesh size is set to unity and the computational domain is
  the given by the number of mesh points defined in {\tt nr}. \\
\begin{code}
    unsigned long int nloc = totalP / Ippl::getNodes();

    partBunch->create(nloc);
    for (unsigned long int i = 0; i< nloc; i++) {
        for (int d = 0; d<Dim; d++)
            partBunch->R[i](d) =  IpplRandom() * nr[d];
    }

    partBunch->qm =  1.0/totalP;
    partBunch->myUpdate();
    partBunch->initFields();
\end{code}
  Now each node created {\tt nloc} particles and initialized the coordinates randomly in the
  computational domain.  A fixed charge to mass ration is assigned. The \texttt{myUpdate()} moves
  all particles to their node defined by the domain decomposition and initialized the fields. In the last
  call the fields gets initialized with the sinusoidal electric field.
  \clearpage
\begin{code}
    for (unsigned int it=0; it<nt; it++) {

        partBunch->R = partBunch->R + dt * partBunch->P;
        partBunch->myUpdate();
        partBunch->gather();
        partBunch->P += dt * partBunch->qm * partBunch->E;
    }
    return 0;
}
\end{code}
The last part of main consists of a simple integration scheme to advance the particles. The call
{\tt gather} interpolates the electric field at the particle position form the nearby grid points by a second
order {\it cloud in cell} (CIC) interpolation scheme.


\subsection{\deprecated{{\em initFields}}}

\begin{code}
void initFields() {

    NDIndex<Dim> domain = getFieldLayout().getDomain();

    for(int i=0; i<Dim; i++)
        nr_m[i] = domain[i].length();

    int nx = nr_m[0]; int ny = nr_m[1]; int nz = nr_m[2];

    double phi0 = 0.1*nx;

    Index I(nx), J(ny), K(nz);

    assign(EFD_m[I][J][K](0),
            -2.0*pi*phi0/nx *
            cos(2.0*pi*(I+0.5)/nx) *
            cos(4.0*pi*(J+0.5)/ny) * cos(pi*(K+0.5)/nz));

    assign(EFD_m[I][J][K](1),  ..... ;

    assign(EFD_m[I][J][K](2),  ..... ;

    assign(EFDMag_m[I][J][K],
           EFD_m[I][J][K](0) * EFD_m[I][J][K](0) +
           EFD_m[I][J][K](1) * EFD_m[I][J][K](1) +
           EFD_m[I][J][K](2) * EFD_m[I][J][K](2));
}
\end{code}

\subsection{\deprecated{{\em myUpdate}}}

 \begin{code}
void myUpdate() {

    if(fieldNotInitialized_m) {
         fieldNotInitialized_m=false;
         getMesh().set_meshSpacing(&(hr_m[0]));
         getMesh().set_origin(rmin_m);
         EFD_m.initialize(getMesh(), getFieldLayout(), GuardCellSizes<Dim>(1), vbc_m);
         EFDMag_m.initialize(getMesh(), getFieldLayout(), GuardCellSizes<Dim>(1), bc_m);
    }
    this->update();
}
\end{code}

  \subsection{\deprecated{{\em gather}}}

\begin{code}
void gather() {
    IntCIC myinterp;
    E.gather(EFD_m, this->R, myinterp);
}
\end{code}

\section{\deprecated{Installation}}
 \ippl uses the {\em cmake } build philosophy.  The following environment variables must be set
\begin{verbatim}
IPPL_ROOT
\end{verbatim}
defining where \ippl is installed.

\subsection{\deprecated{Building \ippl}}
\begin{verbatim}
cd $IPPL_ROOT
CXX=mpicxx F77=gfortran cmake -DCMAKE_VERBOSE_MAKEFILE=OFF
-DCMAKE_INSTALL_PREFIX=~/extlib/ippl $IPPL_ROOT
\end{verbatim}


\subsection{\deprecated{Used Compilers and Libraries}}
The supported operating systems and libraries are listed in Table \ref{tab:archlib}.
\begin{table}[h]
  \caption{Supported Architectures and needed Libraries}
  \label{tab:archlib}
  \begin{center}
    \begin{tabular}{|lcccc|}
      \hline
      Operating System & HDF5  & H5hut & Compiler & Open MPI\\
      \hline
      Linux (SL) 2.6.18 & hdf5-1.8.5 & V0.99 & GNU 4.4.x, icc11.1 & 1.4.2 \\
      Cray XTx  & hdf5-1.8.5 & V0.99 & GNU 4.4 & - \\
      \hline
    \end{tabular}
  \end{center}
\end{table}





\clearpage
\section{\deprecated{Acknowledgements}}
The contributions of various individuals and groups are acknowledged in the relevant chapters,
however a few individuals have or had considerable influence on the
development, Julian Cummings, Yves Ineichen and Jakob Progsch.
Misprints and obscurity are almost inevitable in a document of this size.
Comments and {\em active contributions}  from readers are therefore most welcome.
They may also be sent to \htmladdnormallink{\texttt{andreas.adelmann@psi.ch}}{mailto:andreas.adelmann@psi.ch}.

\subsection{\deprecated{Citation}}
Please cite \ippl in the following way:
\begin{small}
\begin{verbatim}
@techreport{ippl-User-Guide,
title = "{The IPPL (Independent Parallel Particle Layer)
              Framework }",
author = "A. Adelmann",
institution = "Paul Scherrer Institut",
number = "PSI-PR-09-05",
year = 2009}
\end{verbatim}
\end{small}


%& OPAL \\
% & V 1.0



\chapter{Framework Setup}
\label{sec:setup}

\section{\changed{Initialising \ippl}}
\ippl is initialized by passing \texttt{argc} and \texttt{argv} to \texttt{Ippl()} constructor or by creating an instance of \texttt{Ippl::Options}, configuring it, and then passing that options object to \texttt{Ippl::initialize()}. After the \texttt{Ippl()} constructor call MPI (or any other parallel subsystem) is proper initialized. \\
With \texttt{Ippl::getNodes()} or \texttt{Ippl::myNode()} you can for example gather information how many compute nodes/cores are available and on which of the nodes you are running. \\
\begin{code}
#include "Ippl.h"
int main(int argc, char *argv[])
{
    Ippl ippl(argc,argv);
    .....
\end{code}

\section{ Utility Classes in \ippl}
\ippl provides, and uses internally, a number of useful utility classes which you may find helpful when developing new applications.

\subsection{Inform Class}
The Inform class is used to print messages to the console or to a file. It has an interface which is very similar to the iostream classes in C++, and it is mostly used in those situation where you might print a message to cout or cerr. An Inform object is created with a prefix string, which is then appended to all lines of output from the Inform object. Inform essentially takes in data to be printed, formats it for printing just as an ostream object would, but also appends the prefix message to all lines of output. Most important Inform will also indicate which node printed the message when running in parallel.

\subsubsection{Constructing New Inform Objects}

The constructor for Inform has the form
\begin{smallcode}
Inform(char *prefix = 0, int node = 0)
\end{smallcode}
where prefix is a string to prepend to all output lines, and node indicates on what node the Inform object should actually print out the information it is given. Notice that both of these arguments have default values; if no arguments are used when creating a new Inform object, no prefix will be used, if only one argument is given, then node default to 0, which means this Inform object will only print out messages on node 0.
\begin{smallcode}
Inform blankmsg;
blankmsg << "Some text." << endl;
\end{smallcode}
This Inform object will print the text it is given to standard out. The final "\texttt{endl}" is a special manipulator object, which signals the Inform objec t to print out the message it has been given. It will automatically append an endline to the message if it does not already have one at the end. It is important to use endl with an Inform object if it is not ever used, the Inform object will never print out its accumulated text.
\begin{smallcode}
Inform testmsg("mytest");
testmsg << "More text. argc = " << argc << endl;
\end{smallcode}
Here, the prefix is given, if this is used when running in serial, the output will look like:
\begin{smallcode}
mytest> More text. argc = 1
\end{smallcode}
or, if you use this when there is more than one processor in use, the prefix will also include the node number in curly brackets:
\begin{smallcode}
mytest{0}> More text. argc = 1
\end{smallcode}
On all other nodes than node 0, when this Inform object is used, it will not print out the message.
\begin{smallcode}
Inform testmsg("testall", INFORM_ALL_NODES);
\end{smallcode}
This example is similar to the previous example, except the second argument explicitly specifies which node to print on. This can be a number from 0 .... (num nodes - 1), or, as in this example, it can be \texttt{INFORM\_ALL\_NODES} which indicates the message should be printed on ALL the nodes instead of just one. You can also change the node on which an Inform object will print after it has been created by using the \texttt{setprintNode(int)} method of \texttt{Inform}.

\subsubsection{Predefined Inform Objects}
Creating new Inform objects for printing messages is useful in contexts where you would like a unique prefix to indicate where the message originated, say in a specific class method. However, the \ippl framework provides a set of predefined Inform instances which may be used to quickly generate output message or to make sure all messages have a common prefix. These Inform objects are static members of the IPPL class, which is used to initialize the framework. The predefined instances are:
\begin{smallcode}
IPPL::Info = new Inform ("IPPL") ;
IPPL::Warn = new Inform("Warning");
IPPL::Error = new Inform("Error", INFORM_ALL_NODES);
\end{smallcode}
These three instances are used to print generally informative messages, warning messages, and error messages. \texttt{Info} and \texttt{Warn} only print on node 0 by default; \texttt{Error} will print on all nodes. You may use these to printmessages in your own application:
\begin{smallcode}
*IPPL::Info << "An informative message." << endl;
\end{smallcode}
Notice that here that \texttt{Info} was first dereferenced, since it actually is a pointer to an \texttt{Inform} object. A better (and recommended) way to use these predefined instances is to use a macro which is defined for each instance. The macros to use are \texttt{INFOMSG}, \texttt{WARNMSG}, and \texttt{ERRORMSG}; an example of their use is:
\begin{smallcode}
WARNMSG("rhis is a warning: value = " << warnvalue << endl);
\end{smallcode}
The argument to the macro is then given to the associated \texttt{Inform} object for printing.

\subsection{Timer Class}
Timer is used to perform simple timings within a program for use in, e.g., benchmarking. It tracks real (clock) time elapsed, user time, and system time. It acts essentially as a stopwatch:
initially it is stopped, and YOU tell it to stop and start with method calls. The Timer constructor takes no arguments; you create a new Timer object, and use the following methods:
\begin{smallcode}
//Start the clock running. Time only accumulates in the Timer when it is running.
void start()
void stop()         //Stop the clock. The clock may be started again later.
void clear()        //Resets the accumulated time to zero
float clock_time()  //Reports the accumulated "wall clock" time in seconds.
float user_time()   //Reports theaccurnulated user CPU time in seconds.
float system_time() //Reports the accumulated system CPU time in seconds.
float cpu_time()    //Reports user_time() + system_time()
\end{smallcode}

\clearpage

Example how to use the timer class: \\
\begin{code}

IpplTimings::TimerRef selfFieldTimer_m;    \\ definition

selfFieldTimer_m = IpplTimings::getTimer("computeSelfField");

selfFieldTimer_m.start();
    /* compute something */
selfFieldTimer_m.stop();

IpplTimings::print();

\end{code}



\chapter{\deprecated{FFT}}
\label{sec:fft}

The FFT class provides an interface for performing parallel Fourier transforms of various types on \ippl \texttt{\texttt{Field}} objects. FFT is templated on the type of transform to perform (\texttt{CCTransform}, \texttt{RCTransform}, or \texttt{SineTransform}); the dimensionality \texttt{\texttt{Dim}} of the fields to be transformed, and the floating-point precision type (either \texttt{float} or \texttt{double}). It is capable of transforming along all dimensions of a \texttt{\texttt{Field}} or only specified dimensions, and it handles all of the data transposes required to make the Fourier transforms efficient automatically. The FFT constructor arguments vary slightly depending upon which type of transform you wish to perform. Generally speaking, you provide an {\tt NDIndex} object or objects which contain the domains of the input and/or output \texttt{Field}s for the Fourier transform, an optional array of bools of length \texttt{Dim} indicating which dimensions are to be transformed (default is all dimensions), and an optional bool indicating whether or not to compress the intermediate \texttt{Field}s needed to perform data transposes when they are not in use. The default value of this optional argument is false, but the user can set this argument to true if it is necessary to conserve memory. For a complex--to--complex Fourier transform, the input and output fields are of the same element type and are the same size, so only one domain argument is needed. So in the simple case of transforming all dimensions of a \texttt{Field} of type {\tt complex<double>}, we would construct the FFT object with the code
\begin{smallcode}
FFT<CCTransform,Dim,double> ccfft(domain);
\end{smallcode}
where domain is an {\tt NDIndex<Dim>} describing the domain of complex \texttt{Field}s to be transformed with the FFT object.
A real--to--complex Fourier transform takes a field of real numbers and returns a field of complex numbers (or vice-versa for an inverse complex--to--real transform), so we require separate domain arguments describing each \texttt{Field} in the FFT constructor. From the theory of Fourier mode analysis, we know that a Fourier transform of $N$ real numbers will produce $N/2+ 1$ unique complex modes, with modes $0$ and $N/2$ being purely real. Some FFT routines take advantage of the fact that if you pack together the real parts of modes $0$ and $N/2$ as one complex number, you can store all the resulting mode information in the same space as required for the input (i.e., $N$ real numbers or $N/2$ complex numbers). Such a technique tends to cause confusion in multidimensional real--to--complex FFTs, since mode data must then be separated out afterwards. So we choose a format in which the $N/2+ 1$ complex modes are stored separately as complex numbers. Thus, when a real--to--complex transform is performed on a {\tt \texttt{Field}} of doubles, the resulting {\tt \texttt{Field}} of type {\tt complex<double>} will have an extent one greater than half the length of the input field along the first dimension to be transformed and the same length along all other dimensions. This conformance of domains is checked by the FFT constructor. We might construct an FFT object for real--to--complex transforms with the line
\begin{smallcode}
FFT<RCTransform,Dim,double> rcfft(rdomain,cdomain,tdim);
\end{smallcode}
where {\tt rdomain} and {\tt cdomain} are the conforming domains for the {\tt real} and {\tt complex} fields and {\tt tdim} is an array of bools indicating whether or not to transform each dimension. Note that we assume the axes of the field are to be transformed along in the order indicated by the domain arguments for a forward FFT and in the reverse order for an inverse FFT. Each \texttt{Index} object inside the provided domain should refer to a particular axis of the input \texttt{Field}, and these axes are transformed along in order. A sine transform is a special type of Fourier transform in which only the sine (odd) modes are retained. This transform has a field of real numbers for both its input and output, and its effect is to keep only that portion of the data which exhibits odd parity (i.e., vanishes at the endpoints of the interval). Typically, one wishes to enforce odd parity along one or more dimensions of a field, and then perform a standard real-to-complex transform along remaining dimensions. Hence, we require that the user provide two arrays of bools in the constructor: the first to indicate along which dimensions to perform a sine transform, and the second to indicate all of the transform dimensions (both sine transforms and standard FFTs). For example,
\begin{smallcode}
FFT<SineTransform,Dim,double> sinefft(rdomain,cdomain,sinedim,tdim);
\end{smallcode}


constructs an FFT object for doing sine transforms along the dimensions indicated by sinedim and a standard real-to-complex FFT over the other dimensions included in tdim. Alternatively, such transforms could be achieved in two steps, doing the sine transforms and the standard FFTs separately. In this case, we might construct our sine transform FFT object with the code
\begin{smallcode}
FFT<SineTransform,Dim,double> sinefft2(rdomain,sinedim);
\end{smallcode}


and then construct a second FFT object for handling the real-to-complex transform. Note that a sine transform FFT object which is doing only sine transforms requires only a single domain argument describing the real input and output \texttt{Field}s in its constructor.

Once the appropriate FFT object has been constructed, a Fourier transform of data is invoked using the transform member function. The normal arguments to this function are an integer value of + 1 or -1 to indicate the sign of the exponential used in the transform (i.e., the direction of the transform, forward or inverse), and the input and output \texttt{Field}s. For this "two-field" form of the transform function, there is also an optional argument of type bool, which indicates whether or not the input \texttt{Field} is considered to be constant by the transform function. The default value of this optional argument is false, which allows the transform routine to attempt to use the input \texttt{Field} as temporary storage and avoid doing an additional data transpose. You should set the value of this argument to true if you must preserve the contents of the input \texttt{Field} for later use. We would use our previously constructed FFT object for real--to--complex transforms to perform a forward FFT in the following manner:
\begin{smallcode}
rcfft.transform(+l,realField,complexField);
\end{smallcode}


The results of the transform are automatically normalized such that a forward transform followed by an inverse transform returns the original data. For convenience, the FFT class has a member function \texttt{setDirectionName} which allows you to associate a character string with each of the transform directions + 1 and -1. You might choose to refer to these directions as "xtok" and "ktox", for example.

In the case of a complex-to-complex FFT or a pure sine transform; the input and output fields are the same size and of the same type. In these instances, we offer the option of performing the transform "in place"; that is, using just one \texttt{Field} argument for both the input and output. For example, we could perform an inverse complex-to-complex FFT with the code
\begin{smallcode}
ccfft.transform(-1,complexField2);
\end{smallcode}



\subsection{Improving FFT Performance}
Some improvement in performance of the transform method may be obtained by careful selection of the axis ordering of input and output \texttt{Field}s. In order to perform a parallel FFT along a particular dimension, the FFT object will first reorder the axes so that the first axis is the one to be transformed. It does this by assigning the field data into a new \texttt{Field} with a domain in which the order of the original \texttt{Index} objects has been permuted. This new \texttt{Field},
which is maintained internally by the FFT class, has a data layout that is serial along this first dimension and parallel along all other dimensions. With this layout, each processor can independently perform FFTs along the serial axis for each of the one-dimensional strips of data it owns. To subsequently transform along another dimension, the FFT object must again transpose the data so that the next dimension to be transformed is now the first dimension and is serial. These data
transposes can be fairly costly to perform. We can eliminate at least one data transpose if the output \texttt{Field} supplied by the user has the same layout characteristics needed for the final transform (or, in the case of an "in place" transform, if the input \texttt{Field} matches the layout needed for either the first or last transform), and has no guard cell layers. For instance, let us assume we have a three-dimensional \texttt{Field} of complex numbers and we want to transform
all dimensions. If the \texttt{Index} objects {\tt I,J}, and {\tt K} describe the first, second, and third axes of our \texttt{Field} domain, we could perform a forward FFT with the line
\begin{smallcode}
ccfft.transform(+l,complexFieldl);
\end{smallcode}

If the first dimension of \texttt{complexField} is serial, the transform method will skip the first data transpose because the input data is already distributed appropriately for transforming along the first dimension. Similarly, if we were to call an inverse transform with this same \texttt{Field}, it would transform the axes in reverse order, and we would be able to skip the final data transpose. Alternatively, we might choose to do this FFT using separate input and output \texttt{Field}s:
\begin{smallcode}
ccfft.transform(+1,complexField1,complexField2);
\end{smallcode}
In this case, the final optional argument to the "two-field" trans form function defaults to false, meaning that \texttt{complexFieldl} is not considered constant and may be used in place of a temporary \texttt{Field} to avoid the first data transpose. In addition, the output \texttt{Field} can be used in place of the final temporary \texttt{Field} if it has the proper layout. If \texttt{complexField2} has its axes reordered so that its first axis is the final axis to be transformed
(e.g., \texttt{K}, then \texttt{I}, then \texttt{J}) and that first axis is serial, then we can skip the final data transpose. This choice of data layout results in a slightly faster parallel FFT, and it is often convenient if all you need to do is transform the data, do a brief computation with the transformed data, and then invert the transform.

Another issue of relevance to the performance of the transform method is the type of routine used to perform the actual one-dimensional FFT. Currently, we provide two options for this. The first is Fortran 77 implementations of FFT routines from the Netlib repository. These are portable and highly optimized routines that we invoke via C++ wrapper functions. The second option (available only on SGI and Cray systems) is native FFT routines from the SGI/Cray Scientific Library. These routines can be substituted for the portable Netlib routines by supplying the option {\tt USE\_SCSL\_FFT} to the configure utility before compiling the \ippl library. These native library routines tend to be somewhat faster than the portable Fortran routines, and we plan to offer the ability to use native FFT routines such as FFTW in the future.

\chapter{\changed{Particles}}
\label{sec:particles}
This section describes the \ippl framework classes which provide the capability to performing particle-based simulations. We first describe how to design and instantiate \texttt{Particle} classes customized to the needs of a specific application, and then discuss the possible operations and expressions in which a particle object may be employed and end with an ready to use example.

\section{Basic \texttt{Particle} Object Characteristics}

The \ippl framework treats \texttt{Particle} classes as containers which store the characteristic data for \texttt{N} individual particles. Each particle has several attributes, such as position, mass, velocity, etc. Looked at in another way, \texttt{Particle} classes store several attribute containers, where each attribute container holds the value of that attribute for all \texttt{N} particles. \texttt{Particle} objects in \ippl may be thought of as shown in the following diagram: ...

There are two particle attributes predefined, namely R (position) and ID a global unique identifier.

The data type of each attribute, the number of attributes, and the names for these attributes are completely customizable, and are specified in the manner described in the following sections. Any number of different \texttt{Particle} classes may be defined and used within a given simulation. Also, the \texttt{Particle} objects may interact with \ippl \texttt{Field} objects or may be used independently. In addition to the attributes, each \texttt{Particle} object uses a specific layout mechanism, which describes the data of the individual particles is spread across the processors in a parallel environment. The \ippl framework provides several different \texttt{Particle} layout classes, any of which may be selected to partition the particle data among processors. The choice of layout depends on the intended use of the \texttt{Particle} object, as discussed later. Once defined and instantiated, \texttt{Particle} objects in the \ippl framework may be used in many ways, including:
\begin{itemize}
    \item Operations involving all the particles within a \texttt{Particle} object may be specified using simple expressions, in a manner very similar to that used for \texttt{Field} objects. These expressions may involve any of the attributes of the particles as well as other scalar data, and they may use not only the standard mathematical operators +, -, *, /, etc., but also standard mathematical functions such a s cos ( ) , exp ( ) , mod ( ) , etc.
    \item Alternatively, you may set up explicit loops that perform operations involving the attributes of a single particle or a subset of all the particles.
    \item \texttt{Particle}s may be created or destroyed during a simulation.
    \item \texttt{Particle}-to-\texttt{Field} and \texttt{Field}-to-\texttt{Particle} operations may be performed (e.g., a particular \texttt{Particle} attribute may be deposited onto a specified \texttt{Field} using a chosen interpolation method).
\end{itemize}

\section{Defining a User-Specified \texttt{Particle} Class}

There is no specific class within the \ippl framework called \texttt{Particle}. Rather, the first step in deploying particles within a \ippl application is to define a user-specified \texttt{Particle} class, which contains the attributes required for each particle, as well as any, specific methods or data the user may need. To do this, the \texttt{ParticleBase} and \texttt{ParticleAttrib} classes are used, along with a selected subclass of the \texttt{ParticleLayout} class. The steps to follow in creating a new \texttt{Particle} class are:
\begin{itemize}
    \item Based on the type of interactions which the particles have with each other and with external objects such as a \texttt{Field}, select a method of distributing the particles among the nodes in a parallel machine.
    \item Next, decide what attributes each particle should possess.
    \item Third, create a subclass of \texttt{ParticleBase} which includes these attributes (specified as instances of the \texttt{ParticleAttrib} class template).
    \item Finally, instantiate this user-defined subclass of \texttt{ParticleBase} and create and initialize storage for the particles which are to be maintained by this object.
\end{itemize}

The following sections describe in more detail how to accomplish these steps.

\subsection{Selecting a Layout: \texttt{ParticleLayout} and Derived Classes}

When used in a parallel environment, the \ippl framework partitions the particles in a \texttt{Particle} container among the separate processors and includes tools to spread the work of computing and the results of expressions involving \texttt{Particle} attributes among the processing nodes. There are, however, different ways in which particles may be distributed among the processors, and the method which should be used depends upon how the particles in a \texttt{Particle}
object will interact with each other and with \texttt{Field} objects (if at all). The \ippl framework includes different \texttt{Particle} layout mechanisms, which are all derived from the \texttt{ParticleLayout} class. Each \texttt{Particle} object needs its own layout object; that is, you cannot create a layout object and give it to more than one \texttt{Particle} object. The methods typically used to determine how to assign particles to particular nodes are based on analysis of the
position (\texttt{R} attribute) of each particle. Thus, \texttt{ParticleLayout} and its derived classes have two template parameters: the type and the dimensionality of the particle position attribute (this particle position attribute is discussed in more detail later). The following sections describe the particle layout mechanisms currently available in the \ippl framework.

\subsection{The \texttt{{\color{red} ParticleUniformLayout} Class}}
{\color{red} can be removed}

\subsection{The \texttt{ParticleSpatialLayout} Class}

\texttt{ParticleSpatialLayout}, in contrast to \texttt{ParticleUniformLayout}, assigns particles to nodes based upon their spatial location relative to a \texttt{FieldLayout}. It is useful when the particles will be interacting with other particles in their neighborhood or with a \texttt{Field} object. \texttt{ParticleSpatialLayout} will keep a particle on the same node as that which contains the section of the \texttt{Field} in which the particle is located. If the particle moves to a new position, this layout will reassign it to a new node when necessary. This will maintain locality between the particles and any \texttt{Field} distributed using this \texttt{FieldLayout}. Further more it will help keep particles which are spatially close to each other local to the same processor as well. As with all the layout classes, \texttt{ParticleSpatialLayout} requires the type and dimensionality of the particle position attribute as template parameters. The constructor for \texttt{ParticleSpatialLayout} takes one argument: a pointer to a \texttt{FieldLayout} object that tells the \texttt{ParticleSpatialLayout} how the \texttt{Field} is allocated among the parallel processors, so that the particles may be maintained local to this \texttt{Field}. Note that you do not, need to create a \texttt{Field} instance itself, you only need to give \texttt{ParticleSpatialLayout} a \texttt{FieldLayout} object. An example of creating an instance of this class is as follows:
\begin{smallcode}
FieldLayout<3> myfieldlayout(Index(l6), Index(16), Index(32));
ParticleSpatialLayout<double,3> myparticlelayout(&myfieldlayout);
\end{smallcode}


Note that the dimensionality of the \texttt{FieldLayout} and the \texttt{ParticleSpatialLayout} (in this example, 3) must be the same. You may also create a \texttt{ParticleSpatialLayout} instance without providing a \texttt{FieldLayout}. In this case, particles will remain on the node on which they were created. If at some future time you wish to provide a \texttt{FieldLayout} object to tell the \texttt{ParticleSpatialLayout} where to place the particles, you may do so using the \texttt{setFieldLayout} (\texttt{FieldLayout<Dim>*}) method of \texttt{ParticleSpatialLayout}. This is useful when reading particles in from an external source and the size of the spatial domain containing the particles is not known until all the particles have been read. The following example demonstrates the use of the capability:
\begin{smallcode}
ParticleSpatiaILayout<double,3> myparticleLayout;
// calculate the size of the domain required to contain all the particles
// create a new FieldLayout object based on these calculations
FieldLayout<3> myfieldlayout(Index(minx, maxx), Index (miny, maxy),
                             Index(minz,maxz);
myparticlelayout.setFieldLayout(&myfieldlayout);
\end{smallcode}

\texttt{ParticleSpatialLayout} also provides functionality to maintain cached ghost particles from neighboring nodes which might be required for particle - particle interaction. A caching policy can be defined using the fourth template parameter of \texttt{ParticleSpatialLayout}:
\begin{smallcode}
typedef UniformCartesian<Dim, double> Mesh_t;
typedef ParticleSpatialLayout<double,Dim,Mesh_t,
        BoxParticleCachingPolicy<double, Dim, Mesh_t> > playout_t;
\end{smallcode}
The available chaching policies are: \texttt{NoParticleCachingPolicy},\texttt{BoxParticleCachingPolicy} and \texttt{CellParticleCachingPolicy}. With \texttt{NoParticleCachingPolicy} there is no caching whatsoever. \texttt{BoxParticleCachingPolicy} extends the interface of \texttt{ParticleSpatialLayout} by two functions \texttt{void setCacheDimension(int d, T length)} and \texttt{void setAllCacheDimensions(T length)} which are used to set the size of the cached region around the local domain in units of space. \texttt{CellParticleCachingPolicy} extends the interface of \texttt{ParticleSpatialLayout} by two functions \texttt{void setCacheCellRange(int d, int length)} and \texttt{void setCacheCellRanges(int d, int length)} which are used to set the size of the cached region around the local domain in units of grid cells of the mesh. \texttt{BoxParticleCachingPolicy} is the default policy.

The caching can be enabled or disabled by calling the \texttt{enableCaching()} or \texttt{disableCaching()} member functions of \texttt{ParticleSpatialLayout}. Caching is disabled by default.


\subsection{Selecting Particle Attributes: The \texttt{ParticleAttrib} \\Class}

\texttt{ParticleAttrib} is a class template that represents a single attribute of the particles in a \texttt{Particle} object. Each \texttt{ParticleAttrib} contains the data for that attribute for all the particles. Within a user-defined \texttt{Particle} class, you declare an instance of \texttt{ParticleAttrib} for each attribute the particles will possess and assigns to it an arbitrary name. \texttt{ParticleAttrib} requires one template parameter, the type of the data for the attribute. As an example, the statement:
\begin{smallcode}
ParticleAttrib<double> density;
\end{smallcode}

declares an instance of \texttt{ParticleAttrib} named 'density', which will store a quantity of type double for all the particles of the \texttt{Particle} class that contains this data member.

\subsection{Specifying a User-Defined \texttt{Particle} Class: \\The \texttt{ParticleBase} Class}

\texttt{ParticleBase} is the class that all user-defined \texttt{Particle} classes must specify as their base class. It stores the list of attributes for the particles (which are maintained as instances of \texttt{ParticleAttrib}) and a selected parallel layout mechanism. In addition to providing all the capabilities for performing operations on the particles and their attributes, \texttt{ParticleBase} also defines two specific attributes which all user-defined \texttt{Particle} classes inherit:
\begin{smallcode}
ParticleAttrib<Vektor<T,Dim>>  R;
ParticleAttrib<unsigned>      ID;
\end{smallcode}


The first attribute, \texttt{R}, represents the position of each particle. Each position is stored as a \texttt{Vektor<T, Dim>}, which is a \ippl data type representing a dim-dimensional vector with elements of type \texttt{T}. The second attribute, \texttt{ID}, stores a unique unsigned integer value for each particle. The values are not guaranteed to be in any particular order, but they are guaranteed to be unique for each particle. \texttt{ParticleBase} has one template parameter, the layout class to be used to assign particles
to processors (e.g., \texttt{ParticleSpatialLayout}). The data type and dimensionality of the particle position attribute (\texttt{R}) will be the same as those used to create the specific \texttt{ParticleLayout} derived class. Each \texttt{ParticleBase} contains one instance of the chosen layout class. There are two constructors for \texttt{ParticleBase}: a default constructor that creates a new instance of the layout class using the layout's default constructor, and a constructor which takes a pointer to an instance of the
layout class. The second version of the \texttt{ParticleBase} constructor is useful when the desired layout class requires arguments to its constructor (e.g., \texttt{ParticleSpatialLayout}, which may be give in a \texttt{FieldLayout} pointer).

Using \texttt{ParticleBase}, \texttt{ParticleAttrib}, and a selected class derived from \texttt{ParticleLayout,} you can create a user-defined \texttt{Particle} class using the following code template: \\
\clearpage
\begin{codeln}
class Bunch : public ParticleBase< ParticleSpatiaILayout<double,3> >
{
public:
    // Attributes for this particle class (besides position and ID).
    ParticleAttrib<double>             qm;      // q/m ratio
    ParticleAttribs Vektor<double,2> > vel; 	// velocity

    // constructor
    Bunch(Layout\_t *L) : ParticleBase<Layout\_t>(L) {
        addAttribute(qm);
        addAttribute(vel);
    }
};
\end{codeln}

Let us describe this example in detail by discussing the important lines in the order of use.

Line 1: You may select whatever name is appropriate for the specialized \texttt{Particle} class, but it must be derived from \texttt{ParticleBase}.

In this case, we explicitly specify the type of layout to use (\texttt{ParticleSpatialLayout}), with particle position attribute type and dimensionality template parameters of double and 3, respectively. Alternatively, \texttt{Bunch} may have been declared as a class template itself and may have passed on the layout template parameters to \texttt{ParticleBase}. In that case, the first line would instead look like
\begin{smallcode}
template <class PLayout>
class Bunch : public ParticleBase<PLayout>
\end{smallcode}

Lines 5-6: Here is where the attributes for the particles in the \texttt{Particle} object are declared. They may be given any name other than \texttt{R} or \texttt{ID}. Instead of stating the type and dimensionality of this attribute specifically, you may also use one of the following typedefs and constants defined in \texttt{ParticleBase}:
\begin{itemize}
          \item \texttt{Dim} - the dimensionality of the particle position attribute (in this example, 3)
          \item \texttt{Position\_t} - the type of data used to store the position attribute components (here, this type is double)
          \item \texttt{Layout\_t} - a synonym for the specified layout class
          \item \texttt{ParticlePos\_t} - a typedef for the particle position attribute; it is shorthand for \texttt{ParticleAttrib< Vektor<Position\_t,Dim> >}
\end{itemize}
and could have been used to specify the attribute \texttt{vel} in the above example as \texttt{ParticlePos\_t vel};
\begin{itemize}
\item \texttt{ParticleIndex\_t} - a typedef for the particle global \texttt{ID} attribute; it is short for \texttt{ParticleAttrib<unsigned>}
\end{itemize}
The constructor for this user-defined class must initialize \texttt{ParticleBase} with a pointer to an instance of the selected layout class.

In this example, the layout class is \texttt{ParticleSpatialLayout}, but using one of the typedefs listed above, we can abbreviate this as \texttt{Layout\_t}. Note that we only define one constructor here, omitting the default constructor. This is done because \texttt{ParticleSpatialLayout} (which we have hard-coded as the layout for this user-defined \texttt{Particle} class) requires an argument to its constructor, and this can only be provided if we use a constructor for our \texttt{Particle} class as shown here. A new instance of this class would be declared in an application as follows:
\begin{smallcode}
Bunch myBunch (new ParticleSpatialLayout<double,3>(myFieldLayout));
\end{smallcode}
where my\texttt{FieldLayout} was a \texttt{FieldLayout} object created previously. The only action that is required in the constructor for the derived class is to inform the base class of the declared attributes,
using the \texttt{addAttribute(.)} method of \texttt{ParticleBase}, which registers the specified \texttt{ParticleAttrib} instance with the parent class \texttt{ParticleBase}. The order in which attributes are registered is not important.

\subsection{Example \texttt{Particle} Classes: {\color{red} The \texttt{Genparticle} and \texttt{GenArrayParticle}} Classes}
{\color{red} can be removed}


\subsection{Using \texttt{Particle} Classes in an Application}

After a specific \texttt{Particle} class has been defined and created in a \ippl application, you may create and initialize new particles, delete unwanted particles, and perform computations involving these particles. This section describes how to accomplish these tasks.

\subsection{Creating New \texttt{Particle}s}

When a \texttt{Particle} object is created, it is initially empty. Storage for new particles is allocated using the create (unsigned) method of \texttt{ParticleBase}. For example, if a \texttt{Particle} object bunch has been created already, the statement
\begin{smallcode}
bunch.create(100);
\end{smallcode}


will allocate storage for 100 new particles. All the attributes for the particles in the \texttt{Particle} object will have this new storage allocated. The data is uninitialized, except for the global \texttt{ID} attribute; you must assign the proper values to the position and any other attributes that have been defined. The new storage is appended to the end of any existing storage.

\texttt{ParticleBase} includes two methods that allow you to query how many particles exist. The function \texttt{getTotalNum()} will return the total number of particles being stored among all the processors; the function \texttt{getLocalNum()} will return the number of particles just on the local node. Although the new storage space is allocated on the local processor on which the call to create was executed, the \texttt{Particle} class will not officially add the particles to its
local count (and will not tell any other processors it has created these new particles) until you call the \texttt{update()} method of \texttt{ParticleBase}. Thus, a call to \texttt{getLocalNum()} will report the same number just before and just after the call to create. The storage does exist after create is called, but only after the \texttt{update} method (which is discussed in more detail in a later section) has been called will all the processors have correct information on their local and total particle counts.

\subsection{Initializing Attribute Data}

After calling create to allocate new storage, you must initialize the data. This should be done after calling create and before calling \texttt{update} for the \texttt{Particle} object. After the data is initialized, the \texttt{update} routine will properly distribute the particles to their correct node based on the layout mechanism chosen for that \texttt{Particle} object and possibly the positions of the particles as set during their initialization. The following example shows one way to initialize the data for newly
created particles when running on a single-processor machine. (This example will be modified in the following section for the case of running in parallel.) \\
\begin{code}
// create and'initialize data for an instance of Bunch
Bunch myBunch(new Bunch::Layout\_t(myFieldLayout));
int currLocalPtcls = myPtcls.getLocalNum();
myBunch.create(100);
for (int i = 0; i < 100; i++) {
    myBunch.R[currLocalPtcls + i] = Vektor<double,3>(0.0, 1.0, 0.0);
    myBunch.vel[currLocalPtcls + i] Vektor<double,3>(1.0, 1.0, 1.0);
}
myBunch.update();
\end{code}


In this example, 100 new particles are created, and the \texttt{R} and \texttt{vel} attributes are initialized to \texttt{Vektor} quantities. Notice that each attribute is accessed simply by specifying it as a data member of the \texttt{myBunch} object. After create was called, even though the 100 particles were not added to the \texttt{Particle} object's count of local particles, the storage was allocated and it was possible to assign values to the new elements in the attribute storage
(accessed simply using the \texttt{[]} indexing operator). Finally, calling \texttt{update} added the new storage to the count, of particles stored in \texttt{myBunch}. Further calls to getLocalNum and getTotalNum would report the proper values.

\subsection{\deprecated{Initializing Attribute Data on Parallel Architectures}}

The code shown in the previous example has one problem when used on parallel architectures: the call to create is performed on each processor, so if there were P processors a total of 100*P particles would be created. This may be the desired behavior, if so, the previous example is sufficient. However, if you are reading data on particle positions and other attributes from a file or some other source, you may wish to create particles on a single processor, and then distribute the
data to the proper nodes. To do this, you need to call create and assign initial data on only one node but call update on all the processors. The \texttt{singlelnitNode()} method of \texttt{ParticleBase} will return a boolean value indicating whether the local processor should be used to create and initialize new particles in this way. The following example demonstrates how to use this method for initializing particles: \\
\begin{code}
// create and'initialize data for an instance of Bunch
Bunch myBunch(new Bunch::Layout\_t(myFieldLayout));
int currLocalPtcls = myPtcls.getLocalNum();
if (myBunch.singleInitNode()) {
    myBunch.create(100);
    for (int i = 0; i < 100; i++) {
        myBunch.R[currLocalPtcls + i] = Vektor<double,3>(0.0, 1.0, 0.0);
        myBunch.vel[currLocalPtcls + i] Vektor<double,3>(1.0, 1.0, 1.0);
    }
}
myBunch.update();
\end{code}


\subsection{Deleting \texttt{Particle}s}

\texttt{Particle}s may also be deleted during a simulation. The method \texttt{destroy (unsigned M, unsigned I)} of \texttt{ParticleBase} will delete \texttt{M} particles, starting with the \texttt{I}th particle. The index \texttt{I} here refers to the local particle index, not the global \texttt{ID} value. Thus \texttt{I = 0} means delete particles starting with the first one on the local processor.

Unlike the situation when creating new particles, the storage locations for the deleted particles will not be removed from attribute data storage until \texttt{update} is called. Instead, the requests to delete particles are cached until the update phase, at which time all the deletions are performed. You are allowed to issue multiple delete requests between \texttt{update}s. For example, if there are 100 particles on a local node, and you request to delete particles 0 to 10
and then request to delete particles 60 to 70, nothing will change in the attribute storage until you call \texttt{update}, and no change will occur to the local and total particle counts until \texttt{update()} is complete.

\subsection{Updating \texttt{Particle}s: The \texttt{update()} Method}

The \texttt{update()} method of \texttt{ParticleBase} is responsible for making sure that all processors have correct information about how many particles exist and where they are located in a parallel machine. As mentioned previously, \texttt{update} must be called by all processors after a sequence of particle creation or deletion operations. The \texttt{update} method is also responsible for maintaining a proper assignment of particles to processors, based on the particular
\texttt{ParticleLayout} class used to create the
\texttt{ParticleBase} object. Typically, this layout mechanism depends on the position of particles, so when particles change their position, they may need to be reassigned to a new processor to maintain the proper layout. In this case, the \texttt{update} method should be called whenever a computation is complete which alters the attributes (e.g. position) that a layout depends upon. The following short example demonstrates using \texttt{update} in conjunction with some operation that alters the x-coordinate of a
set of particles. \\
\begin{code}
// do some computation involving myBunch for several time steps
while (computation_done == false) {
    // for each particle, add some constant to the x coordinate
    myBunch.R(0) += 0.li
    // update the Particle object; this may move particles between nodes
    myBunch.update();
    // determine if the computation is done, etc.
}
\end{code}

\subsection{Using Particle Attributes in Expressions}

Computations involving particle attributes may be performed in many ways. Data-parallel expressions that involve all particles of a given \texttt{Particle} object may be used, or specific loops may be written that employ attribute iterators.


\subsubsection{Attribute Expressions}

Just as with the \texttt{Field} class, you may perform data-parallel operations on particle attributes using a simple expression syntax, which the \ippl framework will translate into efficient inlined code. These operations will be performed for every particle. The expressions may include any of the attributes in a \texttt{Particle} object as well as scalar values, may use mathematical operators such as +, -, *, / etc., and may call standard mathematical functions such as \texttt{cos(
)}, \texttt{exp( )}, \texttt{mod( )} , etc. for an attribute value of each particle. Some examples are shown below.
\begin{smallcode}
double dt = 2.0;
myBunch.R += myBunch.vel* dt;
myBunch. vel = 1. - log (1. + myBunch. R * myBunch. R) ;
myBunch.update();
\end{smallcode}


Attribute expressions will perform their operations on all the particles in the \texttt{Particle} object, including any new particles allocated via a call to create, even before update has been called. This fact is useful when initializing the attributes for newly created particles (e.g., to set the init value for some scalar quantity to zero). Generally, however, unless you are performing an initialization of new particles, you should avoid using particle expressions of this type after calls to create or destroy and before a call to update.

Some attributes, such as \texttt{Vektor}s or \texttt{Tenzor}s, have multiple components, and you may wish to involve only the \texttt{N}th component of the attribute in an expression. To do so, use the \texttt{()} operator to select the \texttt{N}th component of that attribute. For instance, using \texttt{myBunch} from the previous example, you can change just the x-coordinate of the particle position attribute \texttt{R} as follows:
\begin{smallcode}
myBunch.R(0) = myBunch.R(l) - cos(myBunch.R(2));
\end{smallcode}


For 2D or 3D quantities, use two or three indices. For example, if rho is a 3x3 \texttt{Tenzor} attribute of myBunch, you can do the following:
\begin{smallcode}
myBunch.rho(0,0) = -(myBunch.rho(0,l) + myBunch.rho(0,2));
\end{smallcode}

Attribute expressions may also use the where operator in much the same way as for \texttt{Field} expressions. The first argument to where is some expression that results in a \texttt{boolean} value for each particle. The second and third arguments are expressions that will be evaluated for a particle if the first argument is \texttt{true} or \texttt{false}, respectively, for that particle. For example,
\begin{smallcode}
myBunch.vel = where(myBunch.R(0) > 0.0, -2.0 * myBunch.vel, myBunch.vel)
\end{smallcode}

changes the value of the \texttt{vel} attribute in \texttt{myBunch} when the x-coordinate of the particle position is positive.

\subsection{\deprecated{\texttt{Particle} Iterator Loop}}

You also have the capability of performing operations on specific particles using iterators or standard indexing operations. The \texttt{ParticleAttrib} containers in a \texttt{Particle} class may be used just as regular \texttt{STL} containers. The \texttt{begin()} and \texttt{end()} methods of the \texttt{ParticleAttrib} class will return an iterator pointing to the first element and just past the last element, respectively, of the attribute. These iterators may be used in an explicit loop just as if they were pointers into the attribute array.
\begin{smallcode}
ParticleAttrib<unsigned>::iterator idptr, idend = myBunch.ID.end();
for (idptr = myBunch.ID.begin(); idptr != idend; ++idptr)
    cout << "Particle ID value: " << *idptr << endl;
\end{smallcode}


Iterators are available for all \texttt{ParticleAttribs}. As an alternative, you may simply use the \texttt{[]} operator to access the attribute data of the \texttt{N}th particle on a node, treating \texttt{ParticleAttrib} as a regular array of data.
\begin{smallcode}
int nptcls = myBunch.getLocalNum();
for(int i=0; i < nptcls ++i) {
    cout << "Particle ID value: " << myPtcls.ID[i] << endl;
}
\end{smallcode}


\section{{\color{red}Nearest-Neighbor Interactions (Jakob)}}
{\color{red} can be removed}


\subsection{{\color{red}\texttt{Particle} - \texttt{Particle} Interactions}}
{\color{red} can be removed}

\subsection{\changed{\texttt{Particle} - \texttt{Field} Interactions}}

%\section{\texttt{Particle} - \texttt{Field} Interactions}

Many particle-based simulation methods, including "particle-in-cell" (PIC) simulations, rely on the ability of particles to interact with field quantities. For instance, in particle-based accelerator (plasma) simulations, you typically track the motions of charged plasma particles in a combination of externally applied and self-generated electromagnetic fields. In a \ippl application, such fields might be stored as \texttt{Field} objects of type \texttt{Vektor} existing on a pre-defined mesh. Particles moving through this mesh must be able to "gather" the current value of a \texttt{Field} to their exact positions. Additionally, in order to compute the values of self-generated fields, the particles must be able to "scatter" the value of an attribute onto nearby mesh points, producing a \texttt{Field}. These gather/scatter operations are done using a set of \ippl interpolation methods.

\ippl provides a hierarchy of interpolation classes, each derived from the base class \texttt{Interpolate} and each containing the basic \texttt{gather} and \texttt{scatter} functions. The \texttt{gather} method allows you to gather one or more specified \texttt{Field}s into an equal number of \texttt{ParticleAttribs}. Similarly, \texttt{scatter} will accumulate one or more \texttt{ParticleAttribs} on to an equal number of \texttt{Field} objects. An example of how to scatter the particle density to a \texttt{Field} is shown below.
\begin{smallcode}
InterpolateNGP<Dim> mylnterpolater(myBunch);           // create NGP interpolater
Field<double,Dim> ptcl_density(myfieldlayout);         // create density field
myInterpolator.scatter(myBunch.density,ptcl_density);  // do scattet
\end{smallcode}
The various classes derived from \texttt{Interpolate} implement these \texttt{gather} and \texttt{scatter} methods using different well-known interpolation schemes, such as nearest grid point (NGP), linear interpolation, and the subtracted-dipole scheme (SUDS). You may use these provided classes as a template for deriving new classes from \texttt{Interpolate} that implement other interpolation schemes of interest.

In case of the CIC Interpolation and non-cyclic boundary condition, care has to be taken to not place particles in the outer half of boundary cells. Otherwise values will be scattered out of the grid and be irretrivable.

\chapter{Using the \texttt{Field} and Related Classes}

This section introduces the interface of the \texttt{Field} class and related classes. We describe how
to instantiate \texttt{Field} objects, use \texttt{Index} objects to perform index operations, perform expression
operations with overloaded operators, apply boundary conditions, use the where construct for
conditionals, invoke reduction operations, and use mathematical functions.

\section{\texttt{Field} Object Instantiation}

\subsection{\texttt{Field} Template Parameters} \label{sec:tmpl_params}

The \texttt{Field} class is parametrized on 4 template parameters: type \texttt{T}, dimensionality \texttt{Dim},
mesh type \texttt{Mesh}, and centering \texttt{Centering}.
\begin{smallcode}
Field<class T, unsigned Dim, class Mesh=UniformCartesian<Dim,MFLOAT=doub1e>,
      class Centering=Mesh::DefaultCentering>
\end{smallcode}
The \texttt{T} parameter represents the type of data that can be stored inside of a \texttt{Field}. Currently,
the \texttt{Field} class supports the intrinsic types \texttt{bool, int, float, double}. One may use any
user-defined type or class as the template parameter; however, one must also add traits to the
framework to implement the desired data-parallel promotion properties so that \texttt{Field} operations
work. The framework includes
\begin{smallcode}
Vektor<Dim, T>, Tenzor<Dim , T>, SymTenzor<Dim, T>
\end{smallcode}
classes\footnote{The strange spellings avoid conflicts with other classes such as the STL vector class.}
, which are (mathematical) vectors, tensors, and symmetric tensors whose
elements are of type \texttt{T}. Traits are implemented in these classes so that they may serve as elements
of fully-functional \texttt{Field} objects.
The \texttt{Dim} parameter represent the dimensionality of the \texttt{Field} that is being constructed. This
must correspond to the \texttt{Dim} parameters in all other objects used to construct the \texttt{Field}.
The \texttt{Mesh} parameter represents the mesh on which the field is discretized. \ippl
pre-defines two appropriate classes (\texttt{Cartesian} and \texttt{UniformCartesian}) to use for this
parameter, one of which serves as the default value of the \texttt{Field} ``Mesh'' template parameter:
\texttt{UniformCartesian<Dim, double>}. Refer to the \ippl User Refercnce for details on the
\texttt{UniformCartesian} class; basically, it represents a \texttt{Cartesian} mesh with uniform grid spacings.
The \texttt{Cartesian} class represents a cartesian mesh with nonuniform grid spacings. NB.:
the type parameter \texttt{MFLOAT} for \texttt{Cartesian} represents only the data type used to store internal
information like mesh spacing values; if \texttt{double} satisfies the user, he need not specify it.

The \texttt{Centering} parameter represents the centering of the field on its mesh. \ippl
pre-defines \texttt{Cell} and \texttt{Vert} classes to represent cell and vertex centering, and has implementations
of appropriate mechanisms for \texttt{Cartesian} and other classes which use them. \ippl
also predefines a \texttt{CartesianCentering} class to represent more general centerings--combinations
of vertex and cell centering direction-by-direction and component-by-component for
\texttt{Field}s with multicomponent element types such as \texttt{Vektor}. Finally, \ippl predefines a
wrapper class \texttt{CommonCartesianCenterings} with typedef’s several common special
cases to represent face and edge centerings, for example, refer to the \ippl User Reference
for details.

\subsection{Invoking the \texttt{Field} Constructor} \label{sec:inv_field}

There are six steps in the general construction of a \texttt{Field}:
\begin{enumerate}
    \item Construct \texttt{Index} objects, one for each dimension of the \texttt{Field}. The \texttt{Index} objects describe the desired index domain along the axis.
    \item Construct an \texttt{NDIndex} object with the dimensions of the \texttt{Field}. A single \texttt{NDIndex} object contains N \texttt{Index} objects, and fully describes the total index domain.
    \item Populate the \texttt{NDIndex} with the \texttt{Index} objects created in step 1.
    \item Construct a \texttt{FieldLayout} object with the \texttt{NDIndex} object. The \texttt{FieldLayout} object will control how the data of a specified \texttt{Field} object will be partitioned among physical nodes in a parallel environment.
    \item If desired, construct \texttt{BConds} and \texttt{GuardCellSizes} objects for specifying boundary conditions and guard-cell layers, respectively. If unspecified, these default to no-op and zero.
    \item Finally, construct a \texttt{Field} with the \texttt{FieldLayout}, \texttt{BConds}, and \texttt{GuardCellSizes} object as arguments to the constructor. This target \texttt{Field} must be parametrized as described in Section \ref{sec:tmpl_params}. The \texttt{Dim} template parameter must match the one for the \texttt{FieldLayout} and other objects involved, or you will get a compiler error.
\end{enumerate}

For the cases of a 1,2, or 3 dimensional \texttt{Field}, you may omit steps 2 and 3; instead directly pass the one, two, or three \texttt{Index} objects as arguments to the \texttt{FieldLayout()} constructor. The \texttt{Dim} template parameter must match the number of \texttt{Index} objects passed or you will get a compiler error.

The following code segment demonstrates the construction of a single two dimensional \texttt{Field} of \texttt{double}'s using the six-step method described above: \\
\begin{code}
unsigned Dim = 2;
int Nx = 100, Ny = 50;
Index I(Nx), J(Ny)l                 // Step 1
NDIndex<Dim> domain;                // Step 2
domain[0] = I;                      // Step 3
domain[1] = J;                      // Step 3
FieldLayout<Dim> layout(domain);    // Step 4
Field<double, Dim> A(layout);       // Step 5
\end{code}

The following three examples show the construction of a 3 dimensional \texttt{Field} without the intermediate \texttt{NDIndex} construction:
\begin{smallcode}
Index I(100), J(5), K(25);
FieldLayout<3> layout(I,J,K);
Field<double, 3> A(layout);
\end{smallcode}

You may also construct \texttt{Field} via copy constructor, wherein a \texttt{Field} is copied into another \texttt{Field}. This results in an element-by-element copy of the data:
\begin{smallcode}
// assuming we have constructed a 2D Field of doubles in A
A = 2.0;
Field<double, Dim> B(A);
// B now contains the values 2.0 everywhere
\end{smallcode}

\section{The \texttt{Index} Class}

The \texttt{Index} class represents a strided range of indices, and it is used to define the index extent of \texttt{Field} objects on construction and to reference subranges within \texttt{Field}'s in expressions. The constructor for \texttt{Index} takes one, two or three int arguments. In the case of three arguments, these represent the base index value, the bounding index value and the stride. The two and one-argument cases are simplifications, with the one-argument case being qualitatively different; in
particular,
\begin{smallcode}
Index I(8);
\end{smallcode}
instantiates an \texttt{Index} object representing the range of integers from $0$ to $7$ inclusive, with implied stride $1$. The two-argument
\begin{smallcode}
Index J(2,8);
\end{smallcode}
instantiates an \texttt{Index} object representing the range of integers $[2,8]$, with implied stride 1. The three-argument
\begin{smallcode}
Index K(1,8,2);
\end{smallcode}
instantiates an \texttt{Index} object representing the range of integers $[1,8]$, with stride 2; that is the ordered set $\{1,3,5,7\}$.

Note that the single argument in the one-argument case defines the number of elements, rather than the bound. This means that \texttt{Index J(8)}, which represents $[0,7]$, is different than \texttt{Index J(0,8)} and \texttt{Index J(0,8,1)}, which both mean $[0,8]$.

As illustrated in Section \ref{sec:inv_field}, you use \texttt{Index}'s in constructing the \texttt{FieldLayout} object which goes into the \texttt{Field} constructor. The sizes of the \texttt{Index}'s used to construct the \texttt{FieldLayout} determine the size of the \texttt{Field} in each dimension; here size means the number of integers in the range represented by the \texttt{Index}. For example, the following code segment instantiates a 3-dimensional \texttt{Field A} having size 5 in the first dimension, size 9 in the second dimension, and size 4 in the
third dimension: \\
\begin{code}
unsigned Dim = 3;
int Nx = 5;
int Ny = 9;
int Nx = 4;
Index I(Nx), J(Ny), K(Nz);
FieldLayout<Dim> layout(I,J,K);
Field<double, Dim> A(layout);
\end{code}

You can also use \texttt{Index} objects for initializing \texttt{Field} elements with integer ranges of values. This and more typical use of \texttt{Index} object in conjunction with \texttt{Field} objects is discussed in Section \ref{sec:index_fields}.

Finally, \ippl defines various operators on \texttt{Index} objects, mostly used to represent finite-different stencil operations on \texttt{Field}'s, as described in Section \ref{sec:index_fields}. If
\begin{smallcode}
Index I(8);
\end{smallcode}
is an \texttt{Index} object representing $[0,7]$, then the expression
\begin{smallcode}
I - 1
\end{smallcode}
represents the range of the same length offset by $-1$, or $[-1,6]$. Similarly, the expression
\begin{smallcode}
I + 1
\end{smallcode}
represents $[2,8]$.

\section{The \texttt{NDIndex} Class}

The \texttt{NDIndex} class is primarily a container which holds \texttt{N} \texttt{Index} objects. It is templated on the spatial dimension \texttt{N}, and the constructor takes \texttt{N} \texttt{Index} arguments. For example:
\begin{smallcode}
Index I(5), J(9), K(4);
NDIndex<3> Domain(I, J, K);
\end{smallcode}

An \texttt{NDIndex} object appears as an array of \texttt{Index} objects; you may access the \texttt{Index} object for and dimension using the \texttt{[]} operator. For example:
\begin{smallcode}
Index tmpJ = Domain[1];
\end{smallcode}

\section{The \texttt{FieldLayout} Class}

\texttt{FieldLayout} is the class responsible for determining where the data in a \texttt{Field} object is located. It is templated on the number of indices for the \texttt{Field}; when constructing a new \texttt{FieldLayout} object, you must tell it what is the index range for each dimension (or axis). A single \texttt{NDIndex} object may be used as the argument to a new \texttt{FieldLayout} instance:
\begin{smallcode}
Index I(5), J(9), K(4);
NDIndex<3> Domain(I, J, K);
FieldLayout<3> Layout(Domain);
\end{smallcode}
Or, possibly more conveniently, you may just specify the N \texttt{Index} objects to the constructor of the \texttt{FieldLayout} directly, without explicitly creating an \texttt{NDIndex} object:
\begin{smallcode}
Index I(5), J(9), K(4);
FieldLayout<3> Layout(I, J, K);
\end{smallcode}

\subsection{\changed{Specifying Serial or Parallel Layout}}

By default, a \texttt{FieldLayout} object will partition all \texttt{N} dimensions in a parallel fashion. For example a 2D \texttt{Field} with indices running from $0 \ldots 5$ in each dimension, created with a \texttt{FieldLayout} specified as follows:
\begin{smallcode}
Index I(6), J(6);
FieldLayout<3> Layout(I, J);
\end{smallcode}
will have both the \texttt{I} and \texttt{J} indices partitioned across the nodes in parallel. This would lead to a layout something like that shown in the following figure, if there are four nodes:

TODO: BILD

Unless you tell it otherwise, \texttt{FieldLayout} will attempt to distribute the data among the processors by subdividing each dimension in turn until it has the proper number of subregions. Those axes which are considered for subdivision are the \textit{parallel} axes, which means that a given node will only contain \texttt{Field} data for a subset of the indices along that dimension. You can, however, tell \texttt{FieldLayout} which axes to subdivide, and which to maintain as \textit{serial}. Serial axes are
not ever partitioned by \texttt{FieldLayout}. You must have at least on parallel dimension in a given \texttt{FieldLayout}; by default, all axes are parallel.

To specify serial axes, you provide additional arguments to the \texttt{FieldLayout} constructor, using the keywords \texttt{SERIAL} or \texttt{PARALLEL}. If you create a new \texttt{FieldLayout} by just specifying \texttt{N} \texttt{Index} objects, then you may provide up to \texttt{N} more arguments to the constructor to set the corresponding dimension's layout method. For example, we may change the earlier example of a 2D \texttt{Field} to have the second dimension use a serial layout as follows:
\begin{smallcode}
Index I(6), J(6);
FieldLayout<3> Layout(I, J, PARALLEL, SERIAL);
\end{smallcode}
In this case, the data would be partitioned into four subregions like the following (the horizontal direction is the first dimension, the vertical direction the second)

TODO: BILD

If an \texttt{NDIndex} object is used to create the \texttt{FieldLayout} instead of several \texttt{Index} objects, to change the default layout style you must instead provide an array of keywords (of type \texttt{e\_dim\_tag}) specifying the layout for the \texttt{N} dimensions. For example: \\
\begin{code}
Index I(5), J(9), K(4);
NDIndex<3> Domain(I, J, K);
e_dim_tag ParallelMethod[2];
ParallelMethod[0] = PARALLEL;
ParallelMethod[1] = SERIAL;
ParallelMethod[2] = SERIAL;
FieldLayout<3> Layout(Domain, ParallelMethod);
\end{code}

\section{Boundary Condition Classes}

One of the great frustrations in using data parallel objects is the proper representation of boundary conditions. Most data parallel environments (such as HPF or CMFortran) require that one perform special operations to observe periodic or reflected behaviour at the boundary. This requirement obscures the original clarity of the index notation. You construct \texttt{Field} objects within the \ippl framework using \texttt{BConds} boundary condition object which defines the behaviour of \texttt{Field} and \texttt{Field}
indexing operations at the boundaries. This makes the same, clear indexing notation do the right thing under a variety of imposed boundary conditions.

\subsection{Available Boundary Conditions}

\ippl pre-defines classes to represent 6 different forms of boundary conditions:
\begin{enumerate}
    \item Periodic boundary condition: \texttt{PeriodicFace}
    \item {\color{red} Positive reflecting boundary condition: \texttt{PosReflectFace}
    \item Negative reflecting boundary condition: \texttt{NegReflectFace}
    \item Constant boundary condition: \texttt{ConstantFace}}
    \item Zero boundary condition (special case of constant): \texttt{ZeroFace}
    \item {\color{red} Linear extrapolation boundary condition: \texttt{ExtrapolateFace}}
\end{enumerate}
The {\color{red} red} ones are not yet important.


They represent boundary conditions for a single dimension of a (possibly) multidimensional field, for one ``side'' (or face) of the mesh along that dimension. That is, you must specify two boundary condition objects for each dimension of the \texttt{Field} -- one for each face of the mesh along that dimension. These classes are parametrized on the same four template parameters as \texttt{Field} (see Section \ref{sec:tmpl_params}); the defaults for \texttt{Mesh} and \texttt{Centering} are
\texttt{UniformCartesian} and \texttt{Cell}. As a
further refinement, you may specify boundary conditions for individual components of multicomponent \texttt{Field} elements such as \texttt{Vektor}.

\subsection{\changed{Using Boundary Conditions With \texttt{Field}s}}

The \texttt{BConds} class is a container for the individual specialized boundary conditions; this is the argument passed to the \texttt{Field} constructor. A \texttt{BConds} object acts very much like an array of boundary conditions: when first created, the \texttt{BConds} object is empty, and you add new boundary condition objects to it by treating it as a vector and assigning to its elements. The basic procedure is to construct a \texttt{BConds} object, then construct new or use existing boundary condition objects (from the list
above) to fill it, as illustrated in this example: \\
\begin{code}
unsigened Dim = 2;
Index I(4), J(4);
BConds<double, Dim> bc;
bc[0] = new PeriodicFace<double, Dim>(0);
bc[1] = new PeriodicFace<double, Dim>(1);
bc[2] = new PeriodicFace<double, Dim>(2);
bc[3] = new PeriodicFace<double, Dim>(3);
Field Layout<Dim> layout(I, J);
Field <double, Dim> A(layout, GuardCellSizes<Dim>(1), bc);
\end{code}
Again, the individual face boundary condition objects (in this example, \texttt{PeriodicFace}) perform their task for only a single face of the mesh. In this way, there may be different types of boundary conditions in different dimensions. The face boundary-condition constructors take an unsigned argument designating the face according the following numbering convention: The integers 0 and 1 apply to the boundaries of the first coordinate direction where 0 represents the negative face and 1
represents the positive face. The integers 2 and 3 apply to the boundaries of the second coordinate direction where 2 represents the negative face and 3 represents the positive face. This pairing of integers and domains continues into higher dimensions. The constructors also take optional second and third unsigned parameters to specify a single \texttt{Field} element component rather than all of them. Refer to the \ippl User Reference for more details on these classes, and a detailed discussion
of how the various boundary conditions affect \texttt{Field} operations.

\subsection{Default Boundary Condition}

If a \texttt{Field} is constructed with no \texttt{BConds} object specified, the default is for that \texttt{Field} to have NO boundary conditions. In that case, the boundary conditions container within the \texttt{Field} is empty. It is possible to add additional boundary conditions for a specific face to a \texttt{Field} after it has been constructed; to do so, retrieve the boundary condition container from the \texttt{Field} using the method \texttt{Field::getBConds()}, and then add new face-specific boundary conditions to the returned
\texttt{BConds} container object as shown in the previous example.

\section{\deprecated{The \texttt{GuardCellSizes} Class}}

A \texttt{GuardCellSizes} class, an optional argument to the \texttt{Field} constructor, represents the maximum separation (in elements) of \texttt{Field} elements which will be combined in \texttt{Field} expressions. Typically, this reflects the order of finite differencing in stencil operations. The primary reason for guard cells is parallelism -- a \texttt{Field} domain-decomposed into multiple subdomains with data from adjacent subdomains, so that the stencil operations
have all required data locally. The \texttt{GuardCellSizes}
class is parameterized on the unsigned value \texttt{Dim}, which represents the number of dimensions of the \texttt{Field} object. This \texttt{Dim} value must match the corresponding parameter value of the \texttt{Field} object.

The constructor for \texttt{GuardCellSizes} takes either one or two arguments, which are either \texttt{unsigned} or \texttt{unsigned*}. The one-argument forms specify the same number of guard layers for all dimensions; the two-argument forms specify different numbers for the right and left faces; the unsigned forms specify the same number of layers for all dimensions; and the \texttt{unsigned*} forms specify different number of layers for the different dimensions:
\begin{smallcode}
GuardCellSizes(unsigned s);  // Same no. left&right, same for all directions
GuardCellSizes(unsigned *s); // Same no. left&right, value for each direction
// Diff. left&right, same for all directions
GuardCellSizes(unsigned l, unsigned r);
// Diff. left&right, value for each direction
GuardCellSizes(unsigned *l, unsigned *r);
\end{smallcode}

Section \ref{sec:index_fields}, ``Using \texttt{Index} Objects with \texttt{Field}'s'', shows examples using \texttt{Field} indexing to implement stencil operations. It discusses the numbers of guard layers required by one of the examples.

\section{Operations on \texttt{Field} Objects}

\subsection{Assignment}

A single line of code which contains an assignment operator and a \texttt{Field} on the left hand side of the assignment operator is called a \texttt{Field} expression. Many different terms may appear on the right-hand side of a \texttt{Field} expression (or as the second argument in an \texttt{assign()} call as described below). These include scalars, \texttt{Index}'s, \texttt{Field}'s and \texttt{IndexingField}s's. Currently because of the lack of template member functions in
C++ compilers, you must use the \texttt{assign()} function rather than the
\texttt{operator=}:
\begin{smallcode}
assign(Lhs, Rhs);
\end{smallcode}
where \texttt{Lhs} and \texttt{Rhs} are \texttt{Field} expressions. When template member functions become available, you will simply write:
\begin{smallcode}
Lhs = Rhs
\end{smallcode}

Refer to the \ippl Users Reference for more details, and examples showing where you may use \texttt{operator=} and where you must use \texttt{assign()}. The following are examples of legal assignments:\\
\begin{code}
unsigned Dim = 2, int N = 100;
Index I(N), J(N);
FieldLayout<Dim> layout(I, J);
Field<double, Dim> A(layout), B(layout), C(layout);
A = 2.0;
assign(A, 2.0 + B);
assign(B, A + 2.0);
assign(B[I][J], 3.0 + B[I][J]);
assign(A[I][J], I + A[I][J]/C[I][J]);
\end{code}

For cases where more than one term exists on the right hand side of an assignment, the \texttt{assign()} call must be made. Any combination of scalars, \texttt{Field}'s, \texttt{IndexingField}'s (indexed \texttt{Field} objects; see Section \ref{sec:index_fields}), and \texttt{Index}'s can be put as the second argument of the \texttt{assing()} call. The only requirement in combining terms is that the appearance of an \texttt{Index} object anywhere inside of an expression requires that all the \texttt{Field} objects contained in the expression must be indexed. It
is not possible to combine \texttt{Field}'s and \texttt{IndexingField}'s in a single expression. Nor is it possible to combine \texttt{Field}'s and \texttt{Index} objects in a single expression.

Another intermediate solution to accommodate the lack of member function templates (and therefore the ability to use the \texttt{operator=} member function with more than one term on the right hand side of an expression) is the utilization of the accumulation operators (since this does not require member function templates). Thus, instead of writing
\begin{smallcode}
assign(A, 4.0 + B);
\end{smallcode}
one could write
\begin{smallcode}
A = 0.0;
A += 4.0 + B;
\end{smallcode}
This technique can be used with any of the accumulation operators (\texttt{+=,-=,*=,/=}).

\subsection{\deprecated{Using \texttt{Index} Objects with \texttt{Field}'s}} \label{sec:index_fields}

The \texttt{Field} object works intimately with \texttt{Index} objects to perform a wide variety of operations. Use the \texttt{Index} object to specify the access pattern into a data parallel \texttt{Field} object. Do this by using \texttt{Index} objects inside the brackets following a \texttt{Field} object as follows:
\begin{smallcode}
A[I][J] = B[I][J];
\end{smallcode}
A \texttt{Field} object followed by brackets containing \texttt{Index} objects is called an \texttt{IndexingField}, because \ippl internally uses an \texttt{IndexingField} class as the return value for the \texttt{Field::operator[]}.

You can use \texttt{Index} objects to initialize a \texttt{Field} with integer range data -- that is, assign to a strided range of \texttt{Field} elements the values of a strided range of integers multiplied by the element type. This only works of multiplication by an int is defined for the \texttt{Field} element type, which it is for the intrinsic types $\{$\texttt{int, float, double, bool}$\}$ and the \ippl pre-defined \texttt{Field} element classes $\{$\texttt{Vektor, Tenzor,
SymTenzor}$\}$. For multidimensional \texttt{Field}'s, the range of values is replicated along
the other dimensions. For example, given a \texttt{Field} that is size 8 in its first dimension and size 4 in its second dimension, the code segment
\begin{smallcode}
assign(A[I][J], I);
\end{smallcode}
produces the following values in the \texttt{Field A}:
%
   \begin{center}
        \begin{tabular}{|c|c|c|c|c|c|c|c|}
        \hline
        0 & 1 & 2 & 3 & 4 & 5 & 6 & 7 \\
        \hline
        0 & 1 & 2 & 3 & 4 & 5 & 6 & 7 \\        \hline
        0 & 1 & 2 & 3 & 4 & 5 & 6 & 7 \\        \hline
        0 & 1 & 2 & 3 & 4 & 5 & 6 & 7 \\        \hline
        \end{tabular}
   \label{tbl:t1}
   \end{center}
%
(Here, as in subsequent figures like this displaying \texttt{Field} values, the positive direction of the first coordinate is from left to right and the positive direction for the second coordinate is from top to bottom.) Likewise, an assignment of the form
\begin{smallcode}
assign(A[I],[J], J);
\end{smallcode}
produces
%
   \begin{center}
        \begin{tabular}{|c|c|c|c|c|c|c|c|}
        \hline
        0 & 0 & 0 & 0 & 0 & 0 & 0 & 0 \\        \hline
        1 & 1 & 1 & 1 & 1 & 1 & 1 & 1 \\        \hline
        2 & 2 & 2 & 2 & 2 & 2 & 2 & 2 \\        \hline
        3 & 3 & 3 & 3 & 3 & 3 & 3 & 3 \\        \hline
        \end{tabular}
   \label{tbl:t2}
   \end{center}

The \texttt{Index} objects used to access ranges of values in a \texttt{Field} object do not have to be the same \texttt{Index}'s used in constructing the \texttt{FieldLayout} object used to construct the \texttt{Field}. You can use \texttt{Index} objects of smaller size to access a subrange of the \texttt{Field}. For example, if we wanted to have an 8 by 8 \texttt{Field} with zeros everywhere except for a 4 by 4 subregion in the center, the following code segment would accomplish this goal: \\
\begin{code}
unsigned Dim = 2;
Index I(8), J(8);
Index I2(2,5), J2(2,5);
FieldLayout<Dim> layout(I, J);
Field<double, Dim> A(layout);
A = 0.0;
A[I2][J2] = 1.0;
\end{code}
This would produce the following values in the \texttt{Field A}:
%
   \begin{center}
        \begin{tabular}{|c|c|c|c|c|c|c|c|}
        \hline
        0 & 0 & 0 & 0 & 0 & 0 & 0 & 0 \\        \hline
        0 & 0 & 0 & 0 & 0 & 0 & 0 & 0 \\        \hline
        0 & 0 & 1 & 1 & 1 & 1 & 0 & 0 \\        \hline
        0 & 0 & 1 & 1 & 1 & 1 & 0 & 0 \\        \hline
        0 & 0 & 1 & 1 & 1 & 1 & 0 & 0 \\        \hline
        0 & 0 & 1 & 1 & 1 & 1 & 0 & 0 \\        \hline
        0 & 0 & 0 & 0 & 0 & 0 & 0 & 0 \\        \hline
        0 & 0 & 0 & 0 & 0 & 0 & 0 & 0 \\        \hline
        \end{tabular}
   \label{tbl:t2}
   \end{center}

The lower-limiting case range for an \texttt{Index} object is, of course, a single element. For this case you can just use an integer constant or variable; the following assigns a single element of the \texttt{Field A}: \\
\begin{code}
Index I(4), J(4);
FieldLayout<2> layout(I, J);
Field<double,2> A(layout);
A = 0.0;
A[1][1] = 1.0;
\end{code}
The resultant \texttt{Field A} contains the values:
%
   \begin{center}
        \begin{tabular}{|c|c|c|c|}
        \hline
        0 & 0 & 0 & 0 \\        \hline
        0 & 1 & 0 & 0 \\        \hline
        0 & 0 & 0 & 0 \\        \hline
        0 & 0 & 0 & 0 \\        \hline
        \end{tabular}
   \label{tbl:t2}
   \end{center}

The typical use for indexing is stencil operations, using \texttt{Index} expressions adding or subtracting integer constants to represent the finite differences. This amounts to global data transformation upon a \texttt{Field} through the use of \texttt{Index} operations. For example, if a 4 by 4 \texttt{Field} named \texttt{A} is initialized as follows: \\
\begin{code}
unsigned Dim = 2;
int N = 4;
Index I(N), J(N);
FieldLayout<Dim> layout(I, J);
Field<double, Dim> A(layout, GuardCellSizes<Dim>(1));
assign(A[I][J], I + 1);
\end{code}
then the values in the \texttt{Field A} will be:
%
   \begin{center}
        \begin{tabular}{|c|c|c|c|}
        \hline
        1 & 2 & 3 & 4 \\        \hline
        1 & 2 & 3 & 4 \\        \hline
        1 & 2 & 3 & 4 \\        \hline
        1 & 2 & 3 & 4 \\        \hline
        \end{tabular}
   \label{tbl:t2}
   \end{center}

Now, let's form another \texttt{Field}, \texttt{B}, and assign to it the value of an \texttt{IndexingField} of A which represents an indexed operation:
\begin{smallcode}
Field<double, Dim> B(layout);
assign(B[I][J], A[I+1][J]);
\end{smallcode}

Here we see that the \texttt{Field A} has been indexed with something other than a plain \texttt{Index} object. Rather, it has been indexed by an index expression. The \texttt{Index} objects have been overloaded to allow addition and subtraction by integers to produce other \texttt{Index} objects. The framework recognizes this operation as requesting that all the data in \texttt{A} be shifted to the left (along the first dimension in the negative direction) by 1 position. The \texttt{Field B} contains the values:
%
   \begin{center}
        \begin{tabular}{|c|c|c|c|}
        \hline
        2 & 3 & 4 & 0 \\        \hline
        2 & 3 & 4 & 0 \\        \hline
        2 & 3 & 4 & 0 \\        \hline
        2 & 3 & 4 & 0 \\        \hline
        \end{tabular}
   \label{tbl:t2}
   \end{center}

\texttt{Index}ing operations which access data beyond a \texttt{Field} boundary set the target positions to zero. For the reminder of this section, we shall assume this zero valued boundary condition (which is the default condition when no boundary condition is specified). A variety of boundary conditions can be set on each boundary of a \texttt{Field} and are discussed in detail in the next section.

The \texttt{GuardCellSizes} object used to construct \texttt{Field A} in this example must specify at least on guard layer in the 1st dimension, to accommodate the ``$+1$'' in the indexing operation. The one used, \texttt{GuardCellSizes<Dim>(1)} allows ``$+/-1$'' indexing (as in a width-one stencil), and also allows width-one stencils in the 2nd dimension, because the use of the unsigned argument (the constant, 1) specifies one guard layer both left and right for all directions.

Had we wished to shift the \texttt{Field A} down (along the second dimension in the negative direction) we could have written
\begin{smallcode}
assign(B[I][J], A[I][J+1]);
\end{smallcode}
Then the values in the \texttt{Field B} are:
%
   \begin{center}
        \begin{tabular}{|c|c|c|c|}
        \hline
        1 & 2 & 3 & 4 \\        \hline
        1 & 2 & 3 & 4 \\        \hline
        1 & 2 & 3 & 4 \\        \hline
        0 & 0 & 0 & 0 \\        \hline
        \end{tabular}
   \label{tbl:t2}
   \end{center}

You can shift in the positive or negative direction on any \texttt{Index} object used to index a \texttt{Field}. For example, \\
\begin{code}
unsigned Dim = 2;
int N = 4;
Index I(N), J(N);
FieldLayout<Dim> layout(I, J);
Field<double, Dim> A(layout, GuardCellSizes<Dim>(1));
assign(A[I][J], I + J + 1);
\end{code}
will initialize the values in the \texttt{Field A} to:
%
   \begin{center}
        \begin{tabular}{|c|c|c|c|}
        \hline
        1 & 2 & 3 & 4 \\        \hline
        2 & 2 & 4 & 4 \\        \hline
        3 & 4 & 5 & 6 \\        \hline
        4 & 5 & 6 & 7 \\        \hline
        \end{tabular}
   \label{tbl:t2}
   \end{center}
%
and the operation
\begin{smallcode}
Field<double, Dim> B(layout);
assign(B[I][J], A[I+1][J-2]);
\end{smallcode}
will produce a \texttt{Field B} with the values:
%
   \begin{center}
        \begin{tabular}{|c|c|c|c|}
        \hline
        0 & 0 & 0 & 0 \\        \hline
        0 & 0 & 0 & 0 \\        \hline
        3 & 4 & 5 & 0 \\        \hline
        4 & 5 & 6 & 0 \\        \hline
        \end{tabular}
   \end{center}

\subsection{\changed{Overloaded operators}}

\ippl pre-defines a suite of overloaded operators with the \texttt{Field} class. These include the unary $-$ operator; the binary operators, $+$, $-$, $*$, and $/$; and the accumulation operators \texttt{+=, -=, *=, /=}. Traits~\cite{traits} determine the appropriate casts and promotions of mixed types inside \texttt{Field}. For example, a \texttt{Field} of \texttt{int}'s added to a \texttt{Field} of \texttt{double}'s would perform the correct promotion of \texttt{int} to \texttt{double} element by element. As mentioned earlier, the assign operator $=$ does not work
for most cases because of the lack of member function templates. In addition, the relational operators are not directly available due to conflicts in the current HP reference STL implementation. This functionality is provided through the explicit inlined binary function calls:
%
   \begin{center}
        \begin{tabular}{ll}
        \hline
        binary function relationals & corresponding relation operator \\
        \hline
        gt(A, B) & $A > B$ \\
        lt(A, B) & $A < B$ \\
        ge(A, B) & $A >= B$ \\
        le(A, B) & $A  <= B$ \\
        eq(A, B) & $A == B$ \\
        ne(A, B) & $A != B$ \\
        \hline
        \end{tabular}
   \end{center}
%
The return value of these binary relational functions is a conforming \texttt{Field} of \texttt{bool}'s. Here is an example using binary functional relationals in an expression: \\
\begin{code}
unsigned Dim = 2;
Index I(8), J(4);
FieldLayout<Dim> layout(I, J);
Field<double, Dim> A(layout), B(layout), C(layout);
A = 0.0;
assign(B[I][J], J);
assign(C[I][J], I);
assign(A, lt(B, 2,0)*C);
\end{code}
The resulting \texttt{Field A} contains the values
%
   \begin{center}
        \begin{tabular}{|c|c|c|c|c|c|c|c|}
        \hline
        0 & 1 & 2 & 3 & 4 & 5 & 6 & 7\\        \hline
        0 & 1 & 2 & 3 & 4 & 5 & 6 & 7\\        \hline
        0 & 0 & 0 & 0 & 0 & 0 & 0 & 0\\        \hline
        0 & 0 & 0 & 0 & 0 & 0 & 0 & 0\\        \hline
        \end{tabular}
   \end{center}

\subsection{\deprecated{The \texttt{where()} Function}}

Data parallel simulations often require element-by-element conditionals. The \ippl framework provides a \texttt{where} functions which reduces to the inlined conditional \texttt{operator?}: for each element of the \texttt{Field} objects passed to the \texttt{where()} function. The \texttt{where()} function takes three \texttt{Field} arguments:
\begin{smallcode}
assign(A, where(B, C, D));
\end{smallcode}
where the \texttt{Field B}'s a \texttt{Field} of \texttt{bool}'s. The value of \texttt{C} is placed into \texttt{A} everywhere that \texttt{B} is \texttt{true}, and the value of \texttt{D} is placed into \texttt{A} everywhere that \texttt{B} is \texttt{false}. Thus, \\
\begin{code}
unsigned Dim = 2;
Index I(4), J(4);
FieldLayout<Dim> layout(I, J);
Field<double, Dim> A(layout), B(layout), C(layout);
assign(B[I][J], I - 1);
C = 1.0;
assign(A[I][J], where( lt(B, C), B, C));
\end{code}
leaves the following values in \texttt{Field A}:
%
   \begin{center}
        \begin{tabular}{|c|c|c|c|}
        \hline
        -1 & 0 & 1 & 1 \\        \hline
        -1 & 0 & 1 & 1 \\        \hline
        -1 & 0 & 1 & 1 \\        \hline
        -1 & 0 & 1 & 1  \\        \hline
        \end{tabular}
   \end{center}

Since \texttt{where()} returns a \texttt{Field}, invocations of \texttt{where()} may be used as arguments to where(); this allows nested element-by-element conditionals. The following example code\\
\begin{code}
unsigned Dim = 2;
Index I(4), J(4);
FieldLayout<Dim> layout(I, J);
Field<double, Dim> A(layout), B(layout), C(layout), D(layout);
assign(B[I][J], I - 1);
assign(C[I][J], J - 1);
D = 1.0;
assign(A[I][J], where( lt(B, D), B, where( lt(C, D), C, D)));
\end{code}
leaves the following values in \texttt{Field A}:
%
   \begin{center}
        \begin{tabular}{|c|c|c|c|}
        \hline
        -1 & 0 & -1 & -1 \\        \hline
        -1 & 0 & 0 & 0 \\        \hline
        -1 & 0 & 1 & 1 \\        \hline
        -1 & 0 & 1 & 1  \\        \hline
        \end{tabular}
   \end{center}

\subsection{\changed{Mathematical Functions on \texttt{Field}'s} -- (list of functions needs to be checked)}

As would be expected of any framework for scientific simulation, all the standard mathematical operations are included. The unary functions take a \texttt{Field} object and return a \texttt{Field} object of the same dimension and size where the unary operation has been performed upon each element of the \texttt{Field}. The binary functions take two conforming \texttt{Field}'s and apply the function pairwise to each member of the two \texttt{Field}'s to produce a new conforming \texttt{Field} containing the resultant values. The following
functions, mirroring those in math.h, are available in the framework for \texttt{Field} operations:
\begin{smallcode}
acos, asine, atan, cos, sin, tan, cosh, sinh, tanh,
exp, log, log10, pow, sqrt, ceil, fabs, floor.
\end{smallcode}
For machines which provide them in \texttt{math.h}, \ippl provides the \texttt{Field} version of the \texttt{Bessel, gamma, and error} functions
\begin{smallcode}
erf, erfc, gamma, j0, j1, y0, y1.
\end{smallcode}

\subsection{Reduction Operations}

\ippl includes several reduction operations with the \texttt{Field} class. These include determining the maximum and minimum elements of a \texttt{Field}, the global sum and product of all the elements in a \texttt{Field}, and determining the location of the minimum and maximum values within a \texttt{Field} (typically called minloc and maxloc).

\textit{WARNING: Currently, these functions are only implemented on \texttt{Field} objects; they will not work on \texttt{Field} expressions. This means that invocations like \texttt{min(A+B)} and \texttt{min(2.0*A)} are illegal!}

The following exampled code, which demonstrates the usage of these operators,\\
\begin{code}
unsigned Dim = 2;
Index I(10), J(10);
FieldLayout<Dim> layout(I, J);
Field<double, Dim> A(layout);
assign(A[I][J], I + J);
cout << min(A) << endl;
cout << max(A) << endl;
cout << sum(A) << endl;
cout << prod(A) << endl;
\end{code}
produces the following output:
\begin{smallcode}
0
18
900
0
\end{smallcode}

The minloc and maxloc capabilities, rather than being separately named functions, are two-argument forms of the \texttt{min()} and \texttt{max()} functions. The second argument is an \texttt{NDIndex<Dim>} object, which is a multi-dimensional container for \texttt{Index} objects. The minloc and maxloc operations fill an \texttt{NDIndex<Dim>} object with one \texttt{Index} object for each dimension; each \texttt{Index} is of size one, representing a single point. The following code segment demonstrates: \\
\begin{code}
unsigned Dim = 2;
Index I(10), J(10);
FieldLayout<Dim> layout(I, J);
Field<double, Dim> A(layout);
assign(A[I][J], cos((I-2)*(I-2) + (J-2)*(J-2)));
NDIndex<Dim> LocMin, LocMax;
min(A, LocMin);
max(A, LocMax);
\end{code}
The \texttt{NDIndex<Dim>} objects \texttt{LocMin} and \texttt{LocMax} now contain the position (index location) of the minimum and maximum elements of the \texttt{Field} object \texttt{A}.



\appendix
\include{ippl_field_ref}
\include{ippl_index_ref}
\include{ippl_fieldlayout_ref}
\include{ippl_mesh_ref}
\include{ippl_centering_ref}
\include{bibliography}

\printindex

\end{document}
